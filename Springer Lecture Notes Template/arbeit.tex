% This is LLNCS.DEM the demonstration file of
% the LaTeX macro package from Springer-Verlag
% for Lecture Notes in Computer Science,
% version 2.4 for LaTeX2e as of 16. April 2010
%
\documentclass[oribibl]{llncs}
%
\usepackage{makeidx}  % allows for indexgeneration
\usepackage{url}
\usepackage{mhchem}
\bibliographystyle{plain}%{splncs}
%
\begin{document}
%
\frontmatter          % for the preliminaries
%
\pagestyle{headings}  % switches on printing of running heads

%
%\tableofcontents
%
\mainmatter              % start of the contributions
%
\title{Enhancing Sustainable Software Engineering Methods through SECoMo}{}
%
%\titlerunning{Sustainable Software Engineering and SECoMo}  % abbreviated title (for running head)
%                                     also used for the TOC unless
%                                     \toctitle is used
%
\author{Natalie Buchner}
%
%\authorrunning{Ivar Ekeland et al.} % abbreviated author list (for running head)
%
%%%% list of authors for the TOC (use if author list has to be modified)
%\tocauthor{Ivar Ekeland, Roger Temam, Jeffrey Dean, David Grove,
%Craig Chambers, Kim B. Bruce, and Elisa Bertino}
%
\institute{University of Mannheim, Chair of Software Engineering, \\ D -- 68159 Mannheim,\\
%\email{I.Ekeland@princeton.edu},\\ 
%Internet: \texttt{http://swt.informatik.uni-mannheim.de}
}

\maketitle              % typeset the title of the contribution

\begin{abstract}
%(short) motivation of research context
Sustainability is a globally important topic in today's world. It comprises environmental sustainability, but also social and economic issues. Many factors have a negative impact on Sustainability, for example an increase in energy consumption or pollution. The Information and Communication Technology sector contributes to these negative factors as well, for examply by the growing energy consumption of hardware. Software can also have negative impacts on environmental sustainability, thus efforts on increasing software Sustainability are important. The growing field of Sustainable Software Engineering deals with the questions of how to develop sustainable software and how to develop it in a sustainable way. Existing research proposes a number of Sustainability metrics, measurement tools and process or life cycle models, but despite this variety of approaches, it seems that sustainable Software Engineering approaches are not yet well established in practice.
%Work done / Main topic
The Software Eco-Costs Model (SECoMo) approach is a new estimation approach in this field which allows to estimate, calculate and represent the ecological costs of software already from an early stage on in a software project. It enables stakeholders to have an early understanding of the Sustainability impact of a software and to make design decisions accordingly.\\ %Furthermore, SECoMo provides specific metrics for ecological costs that are the basis for the estimation models.\\
The purpose of this seminar thesis is to consider how SECoMo can be integrated with other existing sustainable software engineering approaches and how it can contribute to improving the integration of such approaches in practice: 
%description of work done!
SECoMo enhances the variety of existing approaches by providing a new set of flexible Sustainability metrics and a respective measurement and estimation method, and it can be integrated in existing process and lifecycle models to improve the usefulness of these methods. It enhances the GREENSOFT reference model by providing a concrete approach for including Sustainability aspects early in the development process. Overall, as SECoMo is concrete but very adaptable, it is a suitable approach that can be easily applied and evaluated in a practical context. This can be a starting point for integrating Software Engineering practitioners in the discussions about software sustainability and finally trigger the establishment of sustainable software engineering in practice. 
%metrics: more flexible, cover a broader context

\end{abstract}
%
\newpage
\section{Introduction}
%Introduction into the topic
In 2015, the United Nations (UN) introduced the \textit{Sustainable Development Goals} \cite{UN_transform_15} in order to raise awareness for the concept of Sustainability: they include the fight against poverty, protecting the planet and granting prosperitiy for everybody \cite{nino_sustainable_2017}. The goals emphasize how important the topic of Sustainability has become for the world, and that taking action affects everybody: "governments, the private sector, civil society"\cite[p.\,1]{nino_sustainable_2017} and every private person.\\
Identifying existing negative influences on Sustainability is the first step towards reducing them and revealing the great potential for positive impacts  everybody can achieve. Accordingly, many industries in the private sector  have taken on their responsibility and have started to reduce their negative ecological impacts on Sustainability. The Information and Communication Technology (ICT) sector is no exception: with its products and services it contributes to high power consumption and carbon dioxide emission \cite{gartner07} and thus has a negative effect on global warming \cite{smart2020} - but in the last few years, many efforts have been made to counteract these negative impacts, for example by improving energy efficiency of hardware products or by increasing the Sustainability of data centers \cite{vor2012coolemall}. The focus of these efforts is mainly on reducing the negative impacts of hardware products, but software can have an impact on Sustainability aspects, too - not only by enabling people or processes using it to be more sustainable, but also as a product itself, for example during its development process. %references for that????? keep in mind! 
Efforts directed at improving software Sustainability during its development and usage phases are an important aspect of ICT Sustainability.\\
%%%%% NOW: an overview about what is there?
In the growing research field of Sustainable Software Engineering (SE) there is a great number of scientific contributions covering ideas for example on Sustainability guidelines for software, models for sustainable development life cycles or specific energy measurement tools for certain types of applications. However there is a lack of an evaluation of these approaches in practice and more conrete models and metrics that can be applied in a great variety of software projects are missing.
%Reasons for that might be the fact that it is a research field that has only emerged during the past couple of years and is still in its first steps. %seee paper of SLR etc. % Another aspects though might be that there are still more concrete approaches and metrics missing, that are generic enough to be applied to various software engineering projects, but specific enough to be directly used in practice.
%description of work done (more detailed)
This paper aims at comparing a new approach in the field of Sustainable SE, SECoMo \cite{schulze_cost_2016}, with the existing variety of models and approaches in the field. The goal is to work out how this approach for estimating the ecological costs of a software system can enhance existing approaches for sustainable software and its development, and how it can contribute to increase the practical adoption of Sustainable SE overall.\\
%Structure of the thesis
The relevant background topics for this seminar thesis, Sustainability and its relation to ICT, software in particular, are introduced and defined in the first section. In the following, the related work from the field of Sustainable SE is presented. In addition, the SECoMo approach is introduced and described with a focus on its general purpose and motivation. % and its possible role in a software devleopment project. 
In the following section, SECoMo is considered in the context of Sustainable SE and analyzed regarding its ability to enhance existing approaches, in order to identify its benefits for the research field and its overall impact on the practical adoption. Lastly, limitations of SECoMo in the context of Sustainable SE are presented and a conclusion of the paper is given.

\section{Background}
\subsection{The Concept of Sustainability}
Although it is clear that Sustainability is a concept of increasing importance in the world and impacts many areas of our daily lives, the definition of the term Sustainability itself is not as clear and not always consistent. There are various attempts at understanding and defining the concept of Sustainability, coming from different perspectives which influence the focus of the definition (e.g. the needs of humans or of nature \cite{gladwin_shifting_1995}) and thus it can be hard to find a common understanding \cite{jamieson_sustainability_1998}%(cited?? or just referenced ) 
. In order to successfully work towards a sustainable future, finding this common understanding is absolutely necessary, though \cite{jamieson_sustainability_1998}.\\
Most attempts at defining the concept of Sustainability start with the definition of the concept of \textit{Sustainable Development} in the so-called \textit{Brundtland Report} in 1987: "Sustainable development is development that meets the needs of the present without compromising the ability of future generations to meet their own needs"\cite[p.\,4]{Brundtland1987}. In addition, the requirements for sustainable development according to \cite{Brundtland1987} are related to three different aspects: society, economy and environment. These three aspects are now widely accepted as "the three dimensions of sustainable development" \cite[p.\,1]{UN_transform_15}.\\ %Bild?
Based on this understanding of sustainable development and which dimensions it covers, the concept of Sustainability can be defined as \textit{a holistic concept that embraces environmental, social and economic factors which lead to a decent life for the current generation while maintaining natural, social and economic resources so that future generations are not limited in living the same decent life}. %no cite -> not published. \cite{buchner_sust_16} p.4
%This definition recognizes Sustainability as a concept that enables sustainable development in all its dimensions which a focus on human needs, but not limited to it. It is chosen as the definition of Sustainability that is further referred to in this paper

\subsection{Sustainability and ICT}
%%% The role of ICT and its impacts on sustainability
The ICT sector has a major influence on economies, societies and even people's daily life. It is obvious that ICT plays an important role in the context of Sustainability, too: on one hand, as negative contributor to global warming, due to its increasing carbon dioxide emissions; on the other hand it has a great potential for positive impacts, by reducing the energy consumption of its own products or as an enabler to achieve energy efficiencies in other sectors \cite{smart2020}.\\
The energy consumption of hardware is the main direct impact ICT has on Sustainability. While direct impacts are often quite obvious, indirectly caused effects of ICT (like the ability to enable improvements in other industries) are often overlooked, even though they can have even greater impacts in the long run. A more fine-grained differentiation of the effects of ICT, which is often used in the literature (\cite{hilty_relevance_2006}; \cite{naumann_greensoft_2011}), distinguishes between three types: First-order effects, second-order effects and third-order effects \cite{berkhout_impacts_2001}. While first-order effects refer to direct impacts "of the physical existence of ICT" \cite[p.\,2]{hilty_relevance_2006}, like resource consumption during the production or usage phase of hardware, second-order effects describe rather indirect effects that are caused by using ICT, like the optimization of processes which might lead to resource conservation \cite{naumann_greensoft_2011}. The third-order effects refer to impacts observable on a long-term scale, like the "adaptation of behaviour (e.g. consumption patterns) or economic structures" \cite[p.\,2]{hilty_relevance_2006} which themselves influence different aspects of Sustainability.\\  
%%% Hilty: "`How to use ICT in the service of sustainability"'? %Bild??
If ICT can have negative and positive impacts on Sustainability on so many different levels, the question that comes to mind is: What are concrete approaches so that ICT can actually be used "in the service of sustainability"\cite[p.\,5]{hilty11}? According to Hilty et al. \cite{hilty11}, there are a couple of relevant approaches in the field of ICT that contribute to Sustainability, with Green IT being the most well-known one.
%\\\textbf{Environmental Informatics} deals with applications that process environmental information in order to tackle ecological problems \cite{hilty11}, thus it is related to second-order ICT effects.
%\\\textbf{Sustainable HCI} covers sustainability \textit{in} and \textit{through} software design \cite{hilty11} and is related to first and second-order effects.
Green IT deals with environmental sustainability of ICT products and has been continuously adapted by companies and organizations %references? da war doch was? referenc mit definitionsübersicht?
since the term was coined in a Gartner Report %\textit{Green IT: A New Industry Shock Wave} 
\cite{gartner07} in 2007. It covers the concepts of Green \textit{In} IT and Green \textit{By} IT \cite{hilty11}. While Green In IT refers to ICT as resource consumer and carbon emission producer itself and deals with improving the environmental sustainability of ICT hardware \cite{calero_green_2015}, Green By IT aims at improving the environmental sustainability in related areas (other products or processes) by using ICT as the enabler \cite{hilty11}. Green In IT approaches take impacts of the whole life cycle of ICT products into account, which is important in order to achieve holistic environmental sustainability \cite{hilty11}. Green IT is related to first- and second-order effects of ICT, but as a strategic concept adapted by many organizations, it has the potential to create long-term, third-order effects, too.\\
Even though ICT products consist not only of hardware artifacts but also software artifacts, the main focus of Green In IT is usually on improving the Sustainability of ICT hardware, %see notes for some references here!!
and often Green By IT approaches do not take the impacts of software artifacts fully into account. %reference??

\subsection{Sustainability and Software\label{SustSW}}
%%% Special focus: Software - how to use software in this 
Software artifacts actually play an important role in the relationship of ICT and Sustainability: Although it may seem that software as the immaterial part of ICT products does not have an impact on Sustainability and "is automatically green" \cite[p.\,3]{agarwal_sustainable_2012}, this naive assumption is made too rashly. For example, when software is developed, the process requires a lot of physical resources; when software is used, it can support sustainable causes and projects; %support sustainability for example by processing environmental data to better understand ecolocial challenges; %when software is deactivated, dealing with saving or converting its data might have economic impacts on a company \cite{johann_sustainable_2011}; 
and it can even have third-order effects like changing user behavior towards more sustainable practices (like video conferencing instead of business travel \cite{amsel_toward_2011}). If software is regarded as "the ultimate cause of hardware requirements and energy consumption" \cite[p.\,1]{kern_impacts_2015}, it becomes clear that it is also indirectly influencing energy efficiency aspects.\\
Considering all these direct and indirect impacts, software definitely needs to be taken into account regarding ICT Sustainability considerations. Accordingly, for the ICT subfield of SE there is a need to include Sustainability topics. But so far, such considerations are not a part of traditional SE methods \cite{penzenstadler_supporting_2012}. This is critized by a growing number of authors (\cite{penzenstadler_supporting_2012}; \cite{agarwal_sustainable_2012}; \cite{amsel_toward_2011}), as the awareness for the importance of Sustainability in SE grows. The emerging field of \textit{Green} or (more holistically) \textit{Sustainable Software Engineering} %\footnote{In the following, the term \textit{Sustainable Software Engineering} will be used, as it comprises the aspect of environmental (green) aspects as well as social and economic aspects, which are equally important parts of Sustainability considerations.} %, even though environmental aspects are the majority.} 
aims at filling this gap by providing approaches that deal with sustainable software and its engineering. Sustainable SE thus constitutes another important apporach how to use ICT in the service of Sustainability.\\ %which impacts does it tackle? .. as it relates to first- and second-order impacts of especially Software on Sustainability!
In the literature, there does not yet exist one common definition of Sustainable Software and Sustainable SE \cite{venters_software_2014}. Nevertheless, for the purpose of this paper the following definition of Sustainable Software is chosen, as it best relates to the three dimensions of Sustainability: 
\begin{quote}
	"Sustainable Software is software whose direct and indirect negative impacts on economy, society, human beings, and the environment resulting from development, deployment, and usage of the software is minimal and/or has a positive effect on sustainable development" \cite[p.\,3]{dick_model_2010}
\end{quote}
Accordingly, Sustainable SE provides the systematic methodology to develop Sustainable Software. It is defined by Naumann et al. as follows:
\begin{quote}
	" [...] Sustainable Software Engineering is the art of developing [...] sustainable software with a [...] sustainable software engineering process. Therefore, it is the art of defining and developing software products in a way, so that the negative and positive impacts on sustainable development that result [...] from the software product over its whole life cycle are continuously assessed, documented, and used for a further optimization of the software product" \cite[p.\,3]{naumann_greensoft_2011} %p. 3
\end{quote}
%mention: nice: focus on whole lifecycle -> as already seen with Green IT
This definition provides a holistic description of the idea of Sustainable SE, especially due to its focus on the impacts the software product has over its whole life cycle. %, which is also in line with the concepts for example mentioned for Green In IT.
In this context, Penzenstadler \cite{penzenstadler_supporting_2012} provides a categorization of Software Sustainability aspects that reflects the variety of impacts software can have. It differentiates between four aspects: the development process aspect, the maintenance process aspect, the system production aspect and the system usage aspect \cite{penzenstadler_supporting_2012}. Approaches of Sustainable SE can be related to one or more of these aspects. %The first two aspects represent the development process viewpoint and describe the sustainability of the actual design and development phase (development process aspect) and the sustainability of the subsequent maintenance phase with aspects like monitoring or bug fixing (maintenance process aspect) \cite{penzenstadler_sustainability_2012}. The other two aspects constitute the product viewpoint and describe the sustainability of the actual software product during its production phase with regards to the resource consumption (System production aspect) and the sustainability of using the software product, in terms of indirect impacts triggered by its usage \cite{penzenstadler_sustainability_2012}.\\ %more details: what does Sust mean IN...
%This categorization will be used in the following in order to describe and categorize contributions in the field of Sustainable SE.
%Bild oder Tabelle hier?


\section{Related work in Sustainable SE} 
\subsection{State of Research\label{stateOfResearch}} %The first considerations of Sustainability in combination with software were mentioned by authors in the middle of the 2000s (\cite{seacord_measuring_2003}; \cite{tate_sustainable_2005}), but they mainly considered Sustainability of software in the context of software maintenance and how the longevity of software could be supported, instead of taking environmental and social impacts of Software into account \cite{albertao_measuring_2010}. When the concept of Green IT became popular in the middle of the 2000s and more attention was drawn to the impacts that ICT had on environmental sustainability (\cite{berkhout_impacts_2001}, \cite{hilty_relevance_2006}), the awareness for the role of software specifically grew and was accepted to be "worth a greater attention" \cite[p.\,1]{capra_green_2009}. But as indicated before, neither environmental Sustainability nor Sustainability in general is actually an aspect that is integrated in traditional SE approaches and concrete guidance on how to support it in this field is missing, according to \cite{penzenstadler_supporting_2012}. Consequently, there is a need for the definition of Sustainable Software, guidelines and approaches how to develop and use it and metrics and measurement methods in order to measure and assess its impact.
According to a Systematic Literature Review (SLR) conducted in 2012 \cite{penzenstadler_sustainability_2012}, most of the research activity in the field of Sustainable SE has started in the beginning of the 2010s and has significantly increased since then. This is underlined by the follow-up study from 2014 \cite{penzenstadler_systematic_2014}, which identified around 40 relevant publications that were added just in between the two studies. In summary, the main insights provided by the studies are the following: While there are many contributions over the last years that relate to software and Sustainability in some way, most research still goes into "domain-specific, constructive approaches" \cite[p.\,5]{penzenstadler_sustainability_2012}, and general reference approaches that can be applied to the whole domain of Sustainable SE are missing \cite{penzenstadler_sustainability_2012}. The study from 2014 identified already more contributions with a focus on general aspects of SE processes and Software Design \cite{penzenstadler_systematic_2014}, but also found that most covered application domains are still only remotely connected to SE \cite{penzenstadler_systematic_2014}. Moreover, a lack of case studies and experience reports about applications of proposed approaches and methods in practice was identified \cite{penzenstadler_sustainability_2012}, which shows that the research field is still mostly shaped by academia and the practical view is missing \cite{penzenstadler_systematic_2014}.\\
The following aspects stand out in current research because they are targeted by many authors and are thus the most relevant issues of the field currently:\\
There is a need for a unified definition of the concepts of sustainable software and the role of Sustainability in SE. This need is apparent and mentioned by many authors (\cite{venters_software_2014}; \cite{becker_sustainability_2015}; \cite{calero_green_2015}; \cite{penzenstadler_what_13}). The general opinion is that despite the variety of attempts, it is not possible to identify a commonly accepted definition yet, due to the recency of the research field, but also due to the complex nature of the concept of software sustainability \cite{venters_software_2014}. % all attempts take into account slightly divergent perspectives. %This is a result of the early stage of the research field, the complex nature of the concept of Sustainability and software and the fact that different perspectives exists from which these definitions are created \cite{venters_software_2014}. 
%The criticality of this %"lack of [a] common understanding of the fundamental concepts of sustainability" \cite{becker_sustainability_2015} in the context of software and SE 
This lack of a defined common ground is critical, because until a widely accepted definition is established, most efforts in the field will be likely to "remain insular and isolated" \cite[p.\,5]{venters_software_2014} and thus hinder the practical adoption of Sustainable SE.\\
Integrating Sustainability as a software quality attribute by making it a Non-Functional Requirement (NFR) in official SE standards is emphasized by many authors as well (\cite{penzenstadler_safety_2014}; \cite{amsel_toward_2011}; \cite{agarwal_sustainable_2012}). This step would account for the central role Sustainability plays today \cite{penzenstadler_safety_2014} and help to integrate Sustainability "early in the lifecycle of software development and influence design decisions" \cite[p.\,2]{raturi_developing_2014}, but there are still appropriate analysis and assessment methods missing as well as policies and standards (\cite{penzenstadler_safety_2014}; \cite{venters_software_2014}).\\ %According to \cite{penzenstadler_safety_2014} and \cite{venters_software_2014}, in order to make Sustainability a NFR, there is still a lot to do: there is for example a need for concrete methods to perform sustainability requirements analysis, existing policies and standards need to include sustainability aspects and %generic? / general / robust / widely usable? 
%assessment techniques and sustainability metrics need to be defined in order to enable sustainability assessment.\\ %Existing attempts at providing such methods and metrics will be presented in the next section.\\
%3
In general, as also found by the SLR, the field is missing concrete reference models and methods that can provide holistic frameworks to actually perform Sustainable SE \cite{penzenstadler_sustainability_2012} and thus provide guidance to Sustainable SE researchers and practicioners. %\cite{penzenstadler_sustainability_2012}: "An encompassing reference framework for SE is still missing". %p.6. %even though GREENSOFT was already there & known????
In addition, the need for concrete, unified metrics that enable to measure the actual impact of software on Sustainability is also evident. %cite someone????.
To meet this demand, more and more contributions from the past five to six years aim at providing concrete models and metrics to be used in the context of Sustainable SE. The most relevant ones will be presented in the next section.
%A quite recent contribution that tries to tackle the issue of missing guidance in the field is the \textit{Karlskrona Manifesto for Sustainable Design} \cite{karlskrona} from 2015. This Mainfesto presents "the fundamental principles underpinning design choices that affect sustainability" \cite{becker_sustainability_2015} and with its creation, the authors aimed at finally providing "a common ground and a point of reference"\cite{becker_sustainability_2015} for the research community of sustainable software. With this, a first starting point is given to align efforts in the area of sustainable SE towards a common understanding.   

\subsection{Overview of existing contributions to Sustainable SE Practices\label{overviewContributions}}%%% Overview of what is there
%In the following, some existing approaches of different aspects of Sustainable SE are presented to give an overview about the most relevant suggested practices the field. %They are divided into approaches regarding sustainability metrics and measurement methods, approaches covering software life cycle and process models and approaches that specifically target requirements engineering (RE) %abkürzung!!!%aspects. %, and a some special approaches that tackle specific aspects. %mention best practices? or just add later?
%metrics / measurement methods
\paragraph{Sustainability metrics and measurement methods} Many contributions to Sustainable SE focuses on methods and tools to measure energy consumption of software. %-> aspect software usage!
There are several tools and measurement frameworks for tracking or measuring energy consumption of software, like \textit{GreenTracker} \cite{amsel_toward_2011}, \textit{PowerIndicator} \cite{naumann_how_2008} or \textit{PowerAPI} \cite{noureddine_preliminary_2012}, that all have different focus aspects. PowerAPI for example enables the monitoring of energy consumption of an application during runtime \cite{noureddine_preliminary_2012}. % and especially allows to consider the impact different programming languages have on the power consumption \cite{noureddine_preliminary_2012}. %One of the first tools introduced was , which aims at making users aware of the impacts software has on environmental sustainability by tracking its energy consumption (based on CPU usage) \cite{amsel_toward_2011}. Another tool that also aims at raising awareness for power consumption by software, more specifically websites, is the \textit{Power Indicator} tool by Naumann et al. \cite{naumann_how_2008} - in detail, it depicts if the server of a website is run with renewable energy. %cite again??
%Other approaches like \textit{PowerAPI} by Noureddine et al. \cite{noureddine_preliminary_2012} provide more comprehensive frameworks for measuring energy consumption. PowerAPI in particular enables the monitoring of energy consumption of an application during runtime, and has the specialty to also consider the impact different programming languages have on the power consumption \cite{noureddine_preliminary_2012}.\\
Other approaches introduce mathematical metrics %Bozelli et al. \cite{} %have conducted a SLR that gives an overview about a variety of such metrics %....TODOOOOOO
like the energy efficiency measures proposed by \cite{capra_is_2012} and \cite{johann_how_2012} - they consider the relative energy consumption of applications %(for example in relation to the "useful work done" \cite{johann_how_2012})
and measure it on a low level, which makes it possible to "find resource intensive parts of programs and improve them" \cite[p.\,1]{johann_how_2012}. The required effort to test and use these metrics is quite high \cite{johann_how_2012}, and the metrics are very specific, which indicates that it is not likely that these metrics can be applied in practice soon.\\
Kern et al. \cite{kern_impacts_2015} propose another interesting metric and the respective calculation method for the carbon footprint of a software project. This Sustainable SE contribution from 2015 aims at finally providing a practical approach for measuring the environmental impact of software which can be integrated in a software development process \cite{kern_impacts_2015}. The basic idea is to calculate the respective current footprint of the development process of a software project or of the software product itself \cite{kern_impacts_2015}. The metric itself is expressed as kg \ce{CO2} per person month and is calculated based on rather high-level inputs like number of working days or number of employees \cite{kern_impacts_2015}.\\
Nearly all of the existing metrics and measurement methods focus exclusively on power and energy consumption and thus mainly target environmental sustainability. %also a bit economic as you have to pay for it :D
%mention bell morse? or not? 
Nevertheless, there are efforts to consider software impacts in relation with other dimensions of Sustainability, like the Sustainability properties and related metrics provided by Albertao et al. \cite{albertao_measuring_2010}. 
Based on existing software quality measures, eleven software Sustainability properties were derived, which include aspects like Usability, Efficiency or the Project's Footprint \cite{albertao_measuring_2010}. 
%In addition, this contribution proposes a sustainability measurement method which includes assessing the sustainability metrics after the release of a software in order to derive improvement goals \cite{albertao_measuring_2010}.
%abschlusssatz - so, was ist da jetzt und was fehlt immer noch? - not much on estimation? albertao - late!?


%lifecycle models // process models
\paragraph{Software life cycle, process and other software sustainability models}
A great variety of approaches exist that propose systematic models for Sustainability in the context of Software. Many of them rather target side aspects of Sustainable SE like the ICT impact assessment model \cite{hilty_relevance_2006} or the Sustainability goals reference model \cite{penzenstadler_generic_2013}.
%Such a general model is for example proposed by Hilty et al. \cite{hilty_relevance_2006} with the purpose of assessing the high-level application fields where ICT can have an impact on environmental sustainability. Another model that addresses a quite different aspect is the Sustainability model by Penzenstadler and Femmer \cite{penzenstadler_generic_2013} - it is a reference model that allows to specify Sustainability goals and related values and activities from either a company or product specific point of view \cite{penzenstadler_generic_2013}. Thus, it facilitates the general analysis of sustainability aspects in different contexts, but also enables the definition of sustainability aspects in relation to a software product, for example.\\
 The very important types of systematic models for Sustainable SE are models with a focus on the Software Development Life Cycle (SDLC) and general process models, as they give guidance on how to actually realize the development. Most of these models concentrate on enhancing existing traditional models with activities and best practices that improve the Sustainability of the software development process itself in all its phases. Examples are the models by Agarwal et al. \cite{agarwal_sustainable_2012} and Shenoy and Eeratta \cite{shenoy_green_2011}, which include aspects like simply reducing resource usage like paper \cite{shenoy_green_2011}, but also consider activities like writing energy-efficiency code \cite{agarwal_sustainable_2012}. Furthermore, they emphasize the need to integrate Sustainability considerations in the requirements phase and to use respective measures to ensure Sustainability and software quality (\cite{agarwal_sustainable_2012}; \cite{shenoy_green_2011}).\\ %The conceptual model for sustainable software systems engineering by Betz and Caporale \cite{betz_sustainable_2014} on the other hand has a slightly different focus - it concentrates on integrating the engineering life cycle of software systems with the life cycle of sustainable process engineering to emphasize the importance of both processes in order to tackle Sustainability issues \cite{betz_sustainable_2014}.
%-> mention the scrum process model here, too!? 
The systematic sustainability model that currently seems to be most important %wording???
in research area, as it is highly referenced in many related contributions, is the \textit{GREENSOFT} model \cite{naumann_greensoft_2011}. This "reference model for green and sustainable software and its engineering" \cite[p.\,1]{naumann_greensoft_2011} is a comprehensive model that combines many aspects of Sustainable SE %+ it comprises many previous contributions of this research group
and can be considered to be the best starting point for applying Sustainable SE to practical projects at the moment. It consists of four parts or levels: a \textit{life cycle model} that is used to assess the impact of the software during its different phases; a set of \textit{Sustainability criteria and metrics}; different procedure models (like for development or administration); and a collection of further recommendations for different stakeholders \cite{naumann_greensoft_2011}.
%\begin{itemize}
	%\item a life cycle model of software products; this is used to assess the impact of the software during its different phases like development or usage; \cite{naumann_greensoft_2011}
	%\item a set of sustainability criteria and metrics, based on those proposed by Albertao et al. \cite{albertao_measuring_2010}; \cite{naumann_greensoft_2011}
	%\item different procedure models that specify in more detail how sustainability can be integrated in the different phases of SE (like the development phase), but also related procedures like administration or purchase \cite{naumann_greensoft_2011}
	%\item a collection of further recommendation for the different stakeholders; this includes suggestions of external sources for sustainable programming or the suggestion of a knowledge base for software sustainability best practices \cite{naumann_greensoft_2011}.
%\end{itemize} %%% + picture???
Overall, this complex model aims at providing information on sustainable software, its engineering, use, administration and so on for different stakeholders and is thus a very important contribution to the research field. But nevertheless, it is not yet the perfect model that enables to integrate Sustainability in practical projects right away: The development procedure model for example concentrates on describing how to manage Sustainability %cite from where?????
during the development process (for example with Sustainability reviews \cite{naumann_greensoft_2011}), but it does not specifically say how to realize it with the actual development. For this, the integration of approaches is needed that for example specify how to calculate certain Sustainability metrics or allow the estimation of processing time or energy consumption in early stages \cite{naumann_greensoft_2011}. %check if this cite is correct?!?!?!
%LATER: not now - Mahmoud / Ahmad! or include before! -> is there some critic here???

%%RE!!!
%\paragraph{Sustainability approaches suitable for requirements engineering}
%Recently, the focus on approaches that integrate Sustainability aspects in the subfield of Requirements Engineering has intensified. %quelle dafür????  %mention that this is related to NFR calls etc?
%The Sustainability reference model by \cite{penzenstadler_generic_2013} is a good example for an approach that is specifically targeted for this area - as the authors state, "[i]t is intended to serve as a reference model [...] for a requirements engineer" \cite{penzenstadler_generic_2013} %p.1
%. Further approaches in this area for example go into the direction of creating a "Sustainability Non-Functional Requirements Framework" \cite{raturi_developing_2014} that supports to formulate and classify sustainability requirements. %actung, zitat wörtlich?
%\\
%In addition to these directly related approaches, it is also possible to consider existing approaches from other areas for the field of RE. As mentioned before, for the promotion of Sustainability as NFR respective assessment models and metrics are needed: The proposed sustainability properties and related metrics by Albertao et al. \cite{albertao_measuring_2010} could for example be used here in order to assess sustainability in all its dimensions; if the focus is on environmental sustainability requirements only, also approaches like the carbon footprint calculation method by \cite{kern_impacts_2015} could be used.
%%MORE MORE MORE
%%%TODO - Later
%In conclusion, this overview over some of the most important existing approaches in Sustainable Software Engineering shows 
%most on environmental sustainability! 
% no "`complete"' model or method, metrics so lalala,, often only in late stages -> applied
%not much details about application in practice? (some are tested but not widely) 

%summing up - what of that is used in practice???
\subsection{Adaption of Sustainable SE in practice}%non-academia?
% Moreover, both studies identified a lack of case studies and experience reports about applications of proposed approaches and methods in practice \cite{penzenstadler_sustainability_2012}, which shows that the research field is still mostly shaped by academia and the practical view is missing %actual citation?? --> does it fit??
As the previous sections show, Sustainability in SE is a relevant research field with increasing acadmic interest and many contributions. Nevertheless, as various authors point out, there is a lack of practical evaluation of existing apporaches and of experience reports (\cite{penzenstadler_sustainability_2012}; \cite{penzenstadler_systematic_2014}), and many existing methods and metrics are not yet suitable to be used in practice. %More?? 
This lack of practical evaluation is a sign for the very limited adoption of Sustainable SE practices in non-academic areas. As Kern \cite{kern_how_2016} points out, the field itself is not yet open for non-academic contexts - but this is absolutely relevant in order for Sustainable SE to actually make a difference. Reasons for this problem are on one hand the still missing common ground in the field (see section \ref{stateOfResearch}; \cite{kern_how_2016}) but also the fact that practioners are not yet included in the discussions \cite{kern_how_2016}.\\
Kern proposes ideas how to open up the research field in order to increase the adoption of Sustainable SE practices in the industry and the public sector \cite{kern_how_2016}, and names two possible ways forward: either starting with including software delevopers in the discussion and use them to evaluate existing ideas, or starting with raising awareness for the issue with private users in order to come up with new ways how to characterize sustainable software \cite{kern_how_2016}. The first idea seems to be a logical step forward given the issues of the field, but in the author's opinion, there is still one issue impeding this: As long as there is still a lack of concrete models and metrics that can be easily adopted for a variety of different projects, the adoption and evaluation of existing Sustainable SE approaches will continue to be limited to small, specific cases and will not reach a broader range of practitioners. The existence of reference models like GREENSOFT is already a step in the right direction by providing a framework for actual, sustainble software projects, but it is still missing concrete details for development processes, for example. Thus, what is needed to promote the field of Sustainable SE are such concrete approaches that can still be applied for a variety of projects. %except for some examples like the kern-carbon-footprint approach?
%darauf dann bezug nehmen im secomo teil
%Add -> 12er SMS idea -> what worked for Green IT.. might work here, too? -> or rather in according contributions chapter  "`establishment in practice"' \14er-Study -> there is a lot to do


\section{The SECoMo approach\label{secomo}} % RELATED WORK / FOUNDATION!?
The Software Eco-Cost Model (SECoMo) was introduced by Schulze \cite{schulze_cost_2016} in 2016. This approach provides Software Engineers with generic models and metrics necessary to estimate and express the ecological costs a software system causes when it is used \cite{schulze_cost_2016}. Thus, SECoMo represents a concrete estimation approach for the impact a software system has on environmental sustainability.\\
The main motivation behind SECoMo is to provide an approach that allows to not only measure the ecological costs that are actually caused by a software system, but also to be able to estimate those costs upfront, for example already during the design phase of a software project in order to keep the costs for changes low \cite{schulze_cost_2016}. %%mention here: WHY!!! -> because it is cheaper to make changes here, thus it is better to have more information already in this pahsee!!!
In order to achieve this, SECoMo offers a set of mathematical models which allow to precisely calculate a variety of different eco-cost metrics, based on information that is already available in the design phase: specification models %????? %name the auxiliary models that are created from that already?  %what about the "`generic"' part?
that describe the functionality, behavior and structure of the software system \cite{schulze_cost_2016}. Furthermore, SECoMo is intended to be highly adaptable in order to allow the estimates to be calculated for different levels of details available - an early, an intermediate and an advanced level \cite{schulze_cost_2016}. More information available allow for more accurate estimates to be made \cite{schulze_cost_2016}. %is that okay...? %
In addition to the mathematical models, SECoMo defines a set of eco-cost drivers in order to identify causes for certain ecological impacts and to better describe under which circumstances they occur \cite{schulze_cost_2016}. The auxiliary models used in SECoMo provide information about these cost drivers, but can also be used to express the estimated eco-costs of the software system \cite{schulze_cost_2016}. This way, SECoMo additionally offers a possibility to communicate estimated or measured eco-costs to stakeholders of a project which can use this information to make improved decisions \cite{schulze_cost_2016}.\\ %what about: comprehensible and clear - on the right level!?
%TODO: %mention: how are Eco-Costs defined!? to give context %mention callibration process???
%what benefits does it offer -> pro's / con's, conclusion? - when to use it? - or put that at the end of the chapter? maybe not to much.. or this could be starting point for comparison - rewrite?!
Against this background, the SECoMo approach is intended to be used in the early stages of SE projects to create estimates about the ecological impact of a software system, so as to enable transparency about the Sustainability aspect right from the start \cite{schulze_cost_2016}. This again makes it possible for software engineers and other stakeholders to make decisions about changes to the software at the design stage which take the impact on ecological costs into account \cite{schulze_cost_2016}. 
%- be it to improve certain eco-cost critical aspects of the software because ecological sustainability is a major concern, or to at least be aware of the eco-cost trade-offs other decisions cause that might be motivated by other concerns, e.g. profitability. \cite{schulze_cost_2016}\\ %???? possible?
%+++++++ the aspect that it is CHEAPER!!! to make changes in this phase!!!!!
SECoMo can also be used in the context of defining requirements for a software system, for example in terms of specifying upper bounds for the eco-cost metrics that must not be exceeded, or even to calculate exact eco-costs if enough details are given, for example for the specific usage scenario of a software system %when all hardware details are known?!
\cite{schulze_cost_2016}.\\
%briefly: "`algorithmic"' estimation model? what does it consist of -metrics, (auxiliary models? -> how? not very specific?), & cost-drivers - other order?
%HOW does it work? how to "use" SECoMo? to have a "`process"' - see evaluation chapter
SECoMo can be integrated in a software engineering project as soon as the necessary basis is available: the specification models describing how the software will be structured, which behavior it should follow and what functionalities it should provide. %Ideally, this is reached by using a modeling approach that creates a structural, behavioral and functional view on the modeled software, which are the basis for the further steps of SECoMos estimation technique. SECoMo does not require a specific modeling approach to create these views \cite{schulze_cost_2016}.\\
Then, the following steps of the SECoMo approach can be integrated in any step of the SE process - in the design phase already if early eco-cost estimations are required, otherwise in later iterations of the process to refine the initial estimates or to calculate the actual eco-cost values (for example in the testing phase):\\
\textbf{1. Calibration of the models}, in order to receive accurate values concerning the hardware energy consumption factors (resource factors) that are later needed as inputs to the mathematical models \cite{schulze_cost_2016}.\\ %and resource factors?! -> to implicitly cover these implementation % related things.. somehow
	%so that the information about the hardware aspects is accurate and relates to a given situation %???
\textbf{2. Preparation of the auxiliary models} so that they include all information available regarding the relevant eco-cost drivers (on the intermediate and advanced level certain information can be derived% using Markov Models
	) \cite{schulze_cost_2016}.\\ %e.g. information needed about frequency - of operation, or how often data type is used in memory / persistent?!
\textbf{3. Calculation of the estimates} for the different eco-cost metrics based on the available information %-> so, which level of detail?
	and the specified parameter values, which specify aspects like user type or case \cite{schulze_cost_2016}.
%Based on the estimated or calculated eco-costs it is also possible to derive specific reports on the environmental impact of the software for different types of stakeholders, or to simply communicate the eco-costs via the existing auxiliary models \cite{schulze_cost_2016}. This way, it is possible for stakeholders to make informed decisions on the software design taking the eco-cost information into account, and especially doing so at an early level \cite{schulze_cost_2016}.
% e.g. make decisions, like - oh this data type causes a lot of eco-costs - could be the size of ceratin attributes
% or this scenario is highly (consumative?!) -> we should just decide not to allow certain combinations of actions to be possible



\section{SECoMo in the context of Sustainable SE}
%What is done in the thesis?
\paragraph{Classification of SECoMo in Sustainable SE}
%In order to analyze how SECoMo can contribute to the field of sustainable SE and how it can enhance existing approaches, the following question needs to be answered: How does SECoMo relate to the field of sustainable SE and how can it be classified?\\
The SECoMo approach can be classified as a Sustainable SE approach, as it aligns with many of the basic ideas characterising its definition (see section \ref{SustSW}): SECoMo can be used within a \textbf{software engineering process} to support design decisions; it deals with attributes of \textbf{software products}, namely its ecological costs during the usage phase (part of the software \textbf{life cycle}); as an adaptive estimation and calculation approach it deals with \textbf{continuously assessing and documenting} eco-cost, which are relevant in the context of \textbf{negative and positive impacts on sustainable development}, mainly for environmental sustainability; and it has the general goal of \textbf{optimizing the software product}. While it is not an engineering process itself, it definitely classifies as a method that enhances and supports aspects of such a process with regards to Sustainability.\\
Moreover, SECoMo can be described as an approach that is directed at the impacts of software while it is used. Thus it falls under the category of \textbf{system usage aspect} approaches according to the Software Sustainability aspects categorization \cite{penzenstadler_supporting_2012}. %with regards to the four aspects of Software Sustainability impacts proposed by \cite{penzenstadler_supporting_2012}: Due to its focus on eco-costs of the software product and the fact that those occur whenever the software is used, SECoMo belongs to the approaches that exclusively focus on the \textbf{system usage aspect}. It does not consider eco-costs that appear during the development or maintenance process itself, nor those that are caused by producing the software product. Nevertheless, 
As stated by the author \cite{penzenstadler_what_13}, the system usage aspect is the most relevant one as it has the biggest impacts, especially the more the software is used, thus SECoMo covers a very relevant area of Software Sustainability.\\
With regards to the current research and its most relevant issues, SECoMo can be assessed as follows: The approach itself does not contribute to finding a common definition of Sustainable SE, due to its specific focus on usage eco-costs. It does, however, provide useful metrics and measurement models that can be used for defining Sustainability requirements in a SE process (see section \ref{secomo}; \cite{schulze_cost_2016}), and thus enhances the chances of making Sustainability a NFR in the furture. As for the third aspect of missing concrete reference models and metrics, the next sections will show how SECoMo can contribute to these aspects. 
%With regards to Requirements Engineering, SECoMo does not provide an RE approach per se - but it certainly enables the integration of environmental sustainability aspects in this field, if appropriate elements from the SECoMo approach are used: The existing eco-cost metrics provided by SECoMo are ideal in order to be used as basis to formulate environmental requirements a software must fulfill and to fix required thresholds \cite{schulze_cost_2016}. Moreover, the auxiliary models created during the SECoMo process can be used in order to define eco-cost requirements and in order to communicate them to stakeholders \cite{schulze_cost_2016}.\\
%The actual estimations or calculations derived from the mathematical models of SECoMo on the other hand are adequate assessment measures to check if the defined requirements are fulfilled. This way, SECoMo offers concrete solutions to some of the issues that are present for Sustainability as a (non-functional) requirement (lack of assessment methods for example, as mentioned by \cite{penzenstadler_safety_2014}).\\ %reihct das hier wenn ich den teil vorne rausnehme??
%In conclusion, SECoMo enhances RE approaches in the context of Sustainability with a vocabulary / models to define and communicate requirements and a also a suitable way to assess them.

\paragraph{SECoMo and Sustainability Metrics}
Like the majority of the introduced metrics (see section \ref{overviewContributions}), SECoMo focuses mainly on environmental aspects of Sustainability, the ecological costs of software in particular. The eco-cost metrics that are part of SECoMo range from a very fine-grained level to a high-level scope (e.\,g. from execution costs for a single operation to those for a concrete scenario \cite{schulze_cost_2016}), and are thus suitable for a variety of different purposes for which eco-costs are regarded. Most of the presented metrics and measurements are only usable in a specific context (\cite{naumann_how_2008}; \cite{capra_is_2012}).\\ %no tool, needs to be adapted and done in a software dev process 
What the SECoMo metrics do not provide are information about energy efficiency (like \cite{capra_is_2012} and \cite{johann_how_2012}), and as they are estimated or calculated without information about the concrete implementation, they can not directly be used to identify resource intense coding parts. But the SECoMo metrics can give even more useful insights, as they can be calculated before any implementation is even considered and still give information on which parts of the software might be very resource intense (for example the execution of a certain operation, or updating a certain data type \cite{schulze_cost_2016}).\\
The goal and motivation of SECoMo actually resembles those of the presented carbon footprint calculation method by Kern et al. \cite{kern_impacts_2015}: they share the motivation of providing a concrete method to calculate the environmental impact of a software product, that can be integrated in software development processes in a reasonable way %and thus finally be used in practice??
(\cite{kern_impacts_2015}; \cite{schulze_cost_2016}). Moreover, the authors intend to create awareness and transparency for the aspect of software energy consumption \cite{kern_impacts_2015} (which SECoMo aims at, too, mainly with regards to the stakeholders \cite{schulze_cost_2016}). Finally, Kern et al. emphasize that the benefit of their method lies in its ability to finally integrate Sustainability aspects early in the software development and design process %direct citation????
\cite{kern_impacts_2015} - which is one of the main benefits of SECoMo, too \cite{schulze_cost_2016}.\\ %add as "`late changes are more expensive"' citation from kern-paper here later
The differences between the two approaches lie in the flexibility of the metrics and their scope: The Carbon Footprint metric has a fixed unit and is thus restricted to a certain context, the SECoMo metrics can vary in the units they represent (for example \ce{CO2} emissions, but also dollars \cite{schulze_cost_2016}) and are thus far more flexible. Furthermore, they can adapt to many different contexts due to the variety of aspects from different dimensions they can cover \cite{schulze_cost_2016}, and they are not restricted to the high-level inputs the Carbon Footprint metric requires \cite{kern_impacts_2015}, but can be way more fine-grained \cite{schulze_cost_2016}. In its scoep, SECoMo is more restricted that the Carbon Footprint calculation method: While the latter method can calculate the footprint of the software product itself and in addition the footprint of the development process\cite{kern_impacts_2015}, SECoMo only applies to the eco-costs caused by the software product itself \cite{schulze_cost_2016}. 
%++++++ compare with albertao metrics!!!!
\\ Overall, these differentiation aspects underline the general benefits of the SECoMo eco-cost metrics in relation to existing Sustainability metrics: they are more flexible and can be applied for different contexts and are  thus more adaptive and they allow an early understanding of causes for inefficient energy consumption of a software product, as early as in the design phase. %and most importantly, they represent metrics that do not need to be exactly cal estimations! which none of the other metrics covers??? %check: is that true?
%or only in models part?!

\paragraph{SECoMo and Sustainable SE models}
As outlined in the section \ref{secomo}, SECoMo can be integrated into traditional SE processes by adding the estimation or calculation steps in the design phase or later phases. Thus, it should also be possible to integrate SECoMo into Sustainable SE processes and SDLC models that are enhanced with Sustainability activities (like the ones proposed by \cite{agarwal_sustainable_2012} and \cite{shenoy_green_2011}).\\
A very interesting question is whether it makes sense to \textit{add} SECoMo to the GREENSOFT model. According to \cite{schulze_cost_2016}, SECoMo fits to the GREENSOFT model in general. But as GREENSOFT is a reference model that comprises a variety of approaches \cite{naumann_greensoft_2011}, SECoMo can not simply be integrated \textit{into} the model, but it needs to be considered at which level it makes most sense. The general concept of eco-costs in SECoMo relates to the first level of GREENSOFT, the life cycle model that is used to assess the impacts of the software in different phases \cite{naumann_greensoft_2011} - eco-costs are impacts relevant in the usage phase. The Sustainability Criteria and Metrics as second level of GREENSOFT could be extended by adding SECoMo's eco-cost metrics as part of the environmental metrics. But where SECoMo actually fits best is the third level of GREENSOFT, the procedure models \cite{naumann_greensoft_2011}: For the development procedure model, so far there only exists a process model which suggests ways to manage Sustainability during the development process. What is missing is a concrete method how to include the measurement of Sustainability aspects in this development procedure, and as the authors note themselves, additional aspects like for example early estimations of the energy consumptions may enhance their proposed model \cite{naumann_greensoft_2011}. %p. 7
SECoMo thus is the perfect addition to the development procedure model of GREENSOFT, especially because it enables estimations at such an early stage like the design phase which makes resulting changes way cheaper \cite{schulze_cost_2016}.\\
So overall, SECoMo is a suitable method that can be integrated in existing SDLC models, but also could be added to the GREENSOFT model, as it enhances it by providing a concrete approach not only to measure Sustainability metrics (here: eco-costs), but also to estimate these eco-costs early, before even the first deployable software is available.

\paragraph{Sustainable SE and SECoMo in practice}
The previous paragraphs show that SECoMo can enhance and extend existing approaches of Sustainable SE in many ways. Based on this, SECoMo can also help to facilitate the actual practical adoption of such approaches and Sustainable SE concepts in general.\\
First, by enhancing existing methods, like the GREENSOFT model, or as additional metric next to the Carbon Footprint metric \cite{kern_impacts_2015}, their application in practical cases is more likely as they are more concrete and flexible and it is easier to adopt them.
Furthermore, the SECoMo approach itself is likely to be applied in practical projects due to its clear structure, the high adaptability and its ability to be integrated in existing SE processes. %and its compatibility with existing modelling approaches. 
Thus it provides a suitable starting point for practitioners to start dealing with Sustainable SE topics and for academia to start integrating them in the discussion - a reasonable way to reach the goal of Sustainable SE applied in practice \cite{kern_how_2016}.
%%Kern -> many ideas -> secomo could help with many of them.. 
In addition to integrating practitioners in the discussion, Kern \cite{kern_how_2016} proposes a variety of further ideas how to foster the adoption of Sustainable SE in practice, like formulating concrete rules and guidelines for sustainable software products or integrating Sustainable SE concepts in teaching. SECoMo can be of benefit for these ideas, too, for example by providing the basic metrics on which guidelines are formulated, or as a suitable exemplary approach that can be used in teaching in order to give students an idea about the ecological footprint of software. %%unschön, verbessern?
%\\ Overall, even though sustainable SE is not yet significantly established in practice, the awareness for sustainability in the industry and the development of the research field give hope to expect that this will soon change, just as it did for the field of Green IT \cite{penzenstadler_systematic_2014}.
%finish with penzenstadler 14 -> worked for green IT -> hopefully here too


\section{Limitations} %also in a different dimension? of my paper? add at the end
%There are a few limitations to the benefit of SECoMo for sustainable SE, despite its positive impact on the reserach field overall:\\ %ist nicht das allheilmittel!!!
Being an approach for estimation and calculation of software eco-costs, SECoMo has a focus on environmental sustainability and is thus mainly enhancing software Sustainability in terms of this dimension. It refers to economic sustainability in terms of reducing the costs for late changes to a software product in the development process, or the possibility to express eco-costs in a monetary unit \cite{schulze_cost_2016}, but this is only a minor concern. It could be criticized that in order to be considered a holistically sustainable approach, some kind of social sustainability component would be formally required. %refer to the definition of sustainable software ?! not only "`green"'?
This formal limitation however is qualified %??? relativiert?
by the fact that most existing Sustainable SE approaches actually focus on environmental sustainability, thus SECoMo is not an exception.\\
Another critical aspect is the assumption that SECoMo can have a major impact on establishment of Sustainable SE approaches in practical projects - this is after all mainly based on the concrete nature of the approach itself and the fact that it allows to be used even with only a small amount of input information available \cite{schulze_cost_2016}, which reduces the obstacles for a practical adoption of the method. In order to actually see if this assumption holds true, it is necessary to first perform further evaluation of the SECoMo approach itself.
%LATER!! calibration models!? a) hard??? ungenau b) only implicitly covers impact of programming style, languages, libraries used, dev environments etc -> there could be a great deal of influence (see noureddine -> impact of programming language? and capra -> influence of dev environment?)
%of the integration - how it can't be perfectly integrated?!
%\section{Future work}
%ideas like - include secomo in a "real" project -> how does it work?  
%how realistic is the "calibration" process, or too much overhead?


%The main contributions / findings of the work
\section{Conclusion}
In the context of Sustainability, ICT and especially software play a major role as contributor to negative impacts, but also as enabler for positive impacts. The research field of Sustainable SE deals with approaches to ensure the Sustainability of software products during their whole life cycle. Over the past few years, the amount of contributions to the field has significantly increased, and many approaches exist that tackle different aspects of software sustainability. But a common understanding and concrete guidance is still missing, and despite the growth of the research field, Sustainable SE is still not established in practice.\\
%% SECOMO -> good for practice - very adaptable -> broad range -> concrete relation to widely used modelling approaches, -> set of metrics AND aspect of ESTIMATIOn that is NEW!!!
%With this unique quality the SECoMo approach can also contribute to sustainable SE practices in general. %why??? %first approach that includes ESTIMATION -> true?
The SECoMo approach introduced in this paper is a Sustainable SE approach that focuses on software usage sustainability by estimating or calculating software eco-costs. It has the ability to enhance existing approaches in the field in various ways and might thus be suitable to help leading Sustainable SE to more adaption in practice: With regards to existing metrics, SECoMo provides a new set of eco-cost metrics that can be applied to various types of software development projects as they are more flexible and can cover a broad range of contexts. In comparison to existing Sustainability models in the field, an important role that SECoMo can play is to enhance the already well-rounded GREENSOFT model with a concrete estimation approach for environmental sustainability aspects in order to make its development procedure model more concrete and thus more likely to be adapted in practice. Furthermore, it provides analysis and assessment methods for adding Sustainability as NFR. Over all, SECoMo introduces a new type of approach to Sustainable SE with its ability to estimate eco-costs even before development phase has started. It constitutes a concrete and comprehensive approach that can be applied for a great variety of projects and is thus suitable to be adopted in real life projects and to promote Sustainable SE, which would foster the integration of practitioners into the discussion about Sustainable SE that is so far dominated by academia.\\ %SECoMo can even provide a suitable examplary approach to be integrated into sustainable SE education.\\
%%%%Abstract: As SECoMo can be integrated in all development phases, especially the early ones, it can help to enhance existing life cycle models with a concrete method for understanding and improving ecological sustainability in the design and implementation phases of software engineering. In addition, with its new set of sustainability metrics, SECoMo offers new options for sustainability measurement and assessment in existing models and tools, which base on a general way of software specification that can reasonably be applied in practice.
The future work on SECoMo will show whether it is indeed a suitable approach that can enable the establishment of Sustainable SE in practice. The research field itself needs to open up to practitioners to finally include the practical view on Sustainable SE and to start with evaluating existing approaches in practice, and it needs to work on finding a common understanding of the underlying concepts. But the trend of adopting Green IT and Sustainability concepts in companies shows that these efforts might soon be worthwhile.
% + ausblick -> mehr noch mit KERN ideen auseinander setzen, zB SECOMO & Education? -> etc!!
%ausblick - secomo - needs to be tested in practice -> and in combination! to say more about practice - but good starting point as very good for all kinds of sw projects but also very adaptive for different needs of stakeholders
%LATER also: more work for other dimensions of sutainbility?! - even though according to penzenstadler for example, every other of her 5 dimensions is more covered than environmentall? whatever


\bibliography{lit}

%%
%% ---- Bibliography ----
%%
%\begin{thebibliography}{1}
%%
%\bibitem{schulze_cost_2016}
%Schulze, T. \textit{A Cost model for Expressing and Estimating Ecological Costs of Software-Driven Systems}. PhD thesis, Universit\"at Mannheim, Mannheim, 2016.
%
%%\bibitem {clar:eke}
%%Clarke, F., Ekeland, I.:
%%Nonlinear oscillations and
%%boundary-value problems for Hamiltonian systems.
%%Arch. Rat. Mech. Anal. 78, 315--333 (1982)
%%
%%\bibitem {clar:eke:2}
%%Clarke, F., Ekeland, I.:
%%Solutions p\'{e}riodiques, du
%%p\'{e}riode donn\'{e}e, des \'{e}quations hamiltoniennes.
%%Note CRAS Paris 287, 1013--1015 (1978)
%%
%%\bibitem {mich:tar}
%%Michalek, R., Tarantello, G.:
%%Subharmonic solutions with prescribed minimal
%%period for nonautonomous Hamiltonian systems.
%%J. Diff. Eq. 72, 28--55 (1988)
%%
%%\bibitem {tar}
%%Tarantello, G.:
%%Subharmonic solutions for Hamiltonian
%%systems via a $\bbbz_{p}$ pseudoindex theory.
%%Annali di Matematica Pura (to appear)
%%
%%\bibitem {rab}
%%Rabinowitz, P.:
%%On subharmonic solutions of a Hamiltonian system.
%%Comm. Pure Appl. Math. 33, 609--633 (1980)
%
%\end{thebibliography}

%\clearpage
%\addtocmark[2]{Author Index} % additional numbered TOC entry
%\renewcommand{\indexname}{Author Index}
%\printindex
%\clearpage
%\addtocmark[2]{Subject Index} % additional numbered TOC entry
%\markboth{Subject Index}{Subject Index}
%\renewcommand{\indexname}{Subject Index}
%\input{subjidx.ind}
\end{document}
