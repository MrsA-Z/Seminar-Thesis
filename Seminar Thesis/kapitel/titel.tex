% Die Titelseite der Arbeit

\begin{titlepage}

\begin{center} % zentrieren

  % Logo der Universit{\"a}t Mannheim
  \begin{figure}[ht]
    \centering
    \includegraphics[width=.6\textwidth]{grafiken/unilogo.png}
  \end{figure}
  
  % Vertikaler Zwischenraum
  \bigskip
  \vfill 
  %\begin{framed}
  % Titel der Arbeit und Typ der Arbeit, umrandet
    \begin{center}
     \textsc{{\LARGE Enhancing Sustainable Software Engineering Approaches / Models through SECoMo\\}}
                                % Letztes \\ ist wichtig, beginnt eine neue Zeile f{\"u}r die Art der Arbeit
  
      \bigskip
  
                                % Art der Arbeit, ggf. auszutauschen gegen Seminar- oder Doktorarbeit
      \textbf{Seminar Thesis}
    \end{center}
   % \end{framed}
    \vfill
    \vfill
  
  % Daten des Erstellers, Einreichungsdatum
  % in einer Tabelle ausgerichtet
  \begin{tabular*}{0.62\textwidth}{r@{\extracolsep{\fill}}l}
   submitted: &\ May 2017\\\\
    by: &\ Natalie Buchner\\
		&\ nbuchner@mail.uni-mannheim.de\\
    &\ born July 12th 1993\\
    &\ in Bonn\\
    \\
    Student ID Number: &\ 1496726\\
  \end{tabular*}
  \vfill
  \vfill
  
  % Unten: Kontaktdaten des Lehrstuhls f{\"u}r Softwaretechnik
  
  \rule{\textwidth}{.4pt}\\ % vertikale Linie
  University of Mannheim\\
  Chair of Software Engineering\\
  D -- 68159 Mannheim\\
  Phone: +49 621-181-3912, Fax +49 621-181-3909\\
  Internet: \url{http://swt.informatik.uni-mannheim.de}
\end{center}

\end{titlepage} % Ende des Titelblatts

%%% Local Variables: 
%%% mode: latex
%%% TeX-master: "~/Documents/DA-Vorlage/beispiel/da-beispiel"
%%% End: 
