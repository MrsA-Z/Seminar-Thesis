\subsection{Sustainability and IT}
%%%%%%%%%%%%%%%%%%%%%%%%%somewhere: mention this discrepancy between "`green it"' and sustainability as holisitic concept!! %focus on sustainability holisticly, but in literature often main focus on envrionmental sust, thus "`green IT"'is common term (any references to support that?????)
%%% The role of ICT and its impacts on sustainability
The ICT sector has a major influence on the way economies and societies work, companies deliver their business and the way people live their daily life. As mentioned before, it is obvious that ICT also plays an important role in the context of Sustainability: on one hand, as negative contributor to global warming for example due to its increasing carbon dioxide emissions %cite smart2020 or gartner here?
, or due to the growing problem of e-waste. %source for e-waste?? -> see lund paper
On the other hand it also has a great potential to have positive impacts on Sustainability, by supporting efforts to reduce the energy consumption of its own products for example, but also as an enabler to achieve energy efficiencies in other sectors \cite{smart2020}. % the smart 2020 report (cite?) outlines.. for example different ways (-> see the smart..) p. 9
%describe hardware efforts etc. shortly, idea behing it -> e.g. improving hardware components - or data centers of a company.. or or or -> many papers on that!? % see some papers used for Lund paper?! -> Green IT maturity, strategic aspect etc.

The energy consumption of hardware is one of the main direct impacts ICT has on sustainability, and while these direct impacts are often quite clear, indirectly caused effects of ICT (like the ability to enable improvements in other industries) are often overlooked, even though they can have even greater impacts in the long run. In order to better illustrate the relationship between IT and Sustainability, a more fine-grained differentiation of the effects of IT is often used in the literature,%cite examples or NAME: for example by..
 which distinguishes between three levels: First-order effects, second-order effects and third-order or rebound effects. %todo: \cite{} - correct origin paper - talk about origin? first introduced by... 
%%more later, see notes->
While first-order effects..., second-order effects describe ... .
The third-order effects refer to effects that occur .. %-> rebound theory?!
%mention some examples? - later


%%% Hilty: "`How to use ICT in the service of sustainability"'?
If ICT can have impacts on Sustainability on so many different levels, the question that comes to mind is: How can ICT be used "in the service of Sustainability"\cite{hilty11}?\\
According to Hilty et al. \cite{hilty11}, relevant approaches in the field of ICT that deal with Sustainability in one way or another%or: achieve sustainability or: contribute to sustainability
 are Environmental Informatics, Green IT and Sustainable Human-Computer Interaction (HCI) %abbrev - how to do it?
. %Further approaches could be... Education? Green Labels for hw or sw products, ... %management approaches, etc.?
 Environmental Informatics describes the 

The focus of sustainable hci is on increasing sustainability THROUGH desing -> ... , but also IN Desing ..

The concept of Green IT 
%more details? what is Green IT, mainly green IN and BY it, introduce  -> see hility and other paper!!

%�berleitung: but.. main focus, especially of green it -> is on hardware! but important other aspect of ICT is also: software, the applications!

%%% Special focus: Software - how to use software in this 
%go back - naiive idea - does software even have an impact? -> Yes! relate back to impacts introduction before (see Lund paper here: reuse some stuff?)
%blabla.. describe some examples of impacts to give an idea (see Lund paper?)  why is sustainable software important? and how to observe that -> see Amsel / Argawal! 

%so - also software has an impact - like it in general: as enabler AND as product itself - so its obvious that it should be somehow a topic in the field of software engineering - which deals with the process of creating software (see  a correct definition, maybe this IEEE glossary?)
%but anyway - not really part of "`traditional SE"' (see penzenstadler) - e.g. no NFR (even though required - who said that again? amsel oder argawal?) -> and penzenstadler -> safety security etc -> but not yet %(is that correct????) 
% + mention: in the following: aspects of "`green"' software engineering are (succumbed???) by this defiintion, as the environmental aspect is part of it

%so -> this is the task of the field of sustainable software engineering! -> (how to develop sustainable software and how to develop it in a sustainable way)
%--> coming back to the question: how to use ICT in service of sust? --> sustainable SE should be added as approach! :)

% So - what exactly is it? although many definitions exist and in the reasearch common understanding that NO COMMON definition exists - for this paper relating to Naumann et al (e.g. instead of amsel) -> because it relates to the sustainability dimensions :)
% or in next chapter????


\section{Related work in Sustainable Software Engineering} %Development?
%%% State of research - paragraph?

%start with original considerations, eg. Tate / Seacorde - and the critics, e.g. Penzenstadler - it is no NFR, not integrated in standard SE practices!

% include results of SLR & mapping study - what is there
%many discussions about -> definitions -> e.g. modern tower of babel / green in se / etc. (here or somewhere else?) like before I have the actual definition?
% say that in the recent years more contributions came (short overview in which directions - or already above?) - but also according to studies - what is missing?

% constant convern - definitions not always clear, concepts?! 
% finish with: mention Karlskrona manifesto - aiming at providing common ground etc. -> summarize

%%% Overview of what is there - models etc? - TOMORROW!!!
%general: classification of aspects like penzenstadler to give everything a bit more structure

%lifecycle models

%metrics / measuring (mainly energy efficiency / consumption)

% 


%%% present other areas? what is done there?!
%requirements engineering
%architecture
%design
%quality assurance(?)
%domain-specific frameworks
%or here: eco-label
%green SLA's


