\chapter{Abstract}
%(short) motivation of research context
Sustainability is a central topic governments, businesses and communities globally are dealing with today. Its most prominent aspect is ecological sustainability and the need to fight global warming, but it also concerns social and economic issues. Many factors come into play that have a negative impact on sustainability, for example an increase in energy consumption or pollution. The Information Technology (IT) sector contributes to these negative factors as well, a main reason being the growing energy consumption caused by IT hardware. But software can also have negative impacts on (mainly) ecological sustainability – directly and indirectly. Thus, it is equally important to consider how to increase the sustainability of software.\\
The growing field of sustainable software engineering deals with the questions of how to develop sustainable software and how to develop it in a sustainable way. It covers aspects in all life cycle phases of a software. Existing research proposes a number of sustainability metrics, measurement tools or process models, but despite this variety of approaches, it seems that sustainable software engineering is not yet well established in practice. Possible reasons are the very specific character of most existing tools and measures, and the rather abstract and general character of life cycle models, with concrete methods of calculating and reducing ecological costs missing.\\
%Work done
%Main topic
The Software Eco-Costs Model (SECoMo) approach by Thomas Schulze (2016)is a new estimation approach in this field which allows to estimate the ecological costs of software already from an early stage on in a software project and to represent those costs and their causes in a comprehensible and clear way. With this, it enables stakeholders to have an early understanding of the sustainability impact of a software and to make design decisions accordingly. (Schulze, 2016) %cite?!
\\ %Furthermore, SECoMo provides specific metrics for ecological costs that are the basis for the estimation models.\\
%main analysis
The purpose of this seminar thesis is to consider how SECoMo can be integrated with other existing sustainable software engineering approaches and how it can contribute to improving sustainable software engineering in practice.\\
As SECoMo can be integrated in all development phases, especially the early ones, it can help to enhance existing life cycle models with a specific method for understanding and improving ecological sustainability in the design and implementation phases of software engineering. In addition, with its new set of sustainability metrics, SECoMo offers new options for sustainability measurement in existing models and tools, as they which base on a general way of software specification that can reasonably be applied in practice.