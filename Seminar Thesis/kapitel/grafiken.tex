\chapter{Implementierung}

Zusammen mit dem Betreuer werden use-cases entwickelt anhand deren die
Software programmiert werden soll. Diese dienen auch als Bewertungsgrundlage.\\
\\
Die allgemein empfohlene Verzeichnisstruktur eines Projektes sieht wie folgt
aus:

\begin{itemize}
\item projektname
  \begin{itemize}
  \item bin
  \item doc
  \item lib
  \item src
  \end{itemize}
\end{itemize}

Die zu erstellende Software soll im package
de.uni\_mannheim.informatik.swt.\texttt{projektname} unter \texttt{src} liegen.\\
\\
Bei der Programmierung sollte durchgängig die englische Sprache verwendet
werden. Hierzu zählen insbesondere Kommentare im Quellcode, Namen von
Funktionen, Variable, Menüpunkte im Benutzerinterface, kurze
Hilfestellungen und Ausgaben von Programmen.\\
\\
Der Code sollte mit Hilfe von \texttt{lstinputlisting} formatiert und ausgegeben werden, wie in folgendem Beispiel:

\lstinputlisting[
  language=Java, numbers=left, stepnumber=5, firstnumber=1, breaklines=true, 
  basicstyle=\footnotesize,
  numberstyle=\tiny,
  caption={GuestbookForm.java},
  captionpos=b,
  label=GuestbookForm
]
{code/GuestbookForm.java.txt}
