%Multiple sections, based on what I do
%%%%%%%Contributions}
%\section{Why sustainable software engineering is not yet used in practice(?!)} %see results of SLR?! + paper of Eva Kern
% of existing models / methods / metrics:
% critique - what is missing? - not used in practice?! (quellen?!) - why?

\section{SeCoMo in the context of Sustainable Software Engineering}
%What is done in the thesis?
%%%%Einordnung
\paragraph{Classification}
%how does secomo fit IN Sust. SE - relate it to DEFINITION !
In order to analyze how SECoMo can contribute to the field of Sustainable Software Engineering and how it can enhance existing approaches, the following question needs to be answered: How does SECoMo relate to the field of Sustainable Software Engineering and how can it be classified?\\
Recalling the definition of Sustainable Software Engineering %how to do querverweise -> hier wäre es nett!
 used in this paper, it becomes evident that the SECoMo approach aligns with many of the basic ideas: SECoMo is an approach that can be used within a \textbf{software engineering process} to support design decisions; it deals with attributes of \textbf{software products}, namely its ecological costs during the usage phase (part of the software \textbf{life cycle}); as an adaptive estimation and calculation approach it deals with \textbf{continuously assessing and documenting} eco-cost, which are relevant in the context of \textbf{negative and positive impacts on sustainable development}, mainly for environmental sustainability; and it has the general goal of \textbf{optimizing the software product}.\\
Due to this strong similarities SECoMo can be described as an approach that clearly fits into the field of Sustainable Software Engineering, while not as an engineering process itself, definitely as a method that enhances and supports aspects of such a process with regards to sustainability.\\
More specifically, SECoMo can be classified with regards to the four aspects of Software Sustainability impacts proposed by \cite{penzenstadler_supporting_2012}: Due to its focus on eco-costs of the software product and the fact that those occur whenever the software is used, SECoMo belongs to the approaches that exclusively focus on the \textbf{system usage aspect}. It does not consider eco-costs that appear during the development or maintenance process itself, nor those that are caused by producing the software product. Nevertheless, as stated by \cite{penzenstadler_what_13}, the system usage aspect is the most relevant one as it has the biggest impacts, especially the more the software is used - thus, SECoMo is an approach that is directed at a very important aspect in sustainable software engineering: the sustainability of the software product itself.\\
The next sections deal with the question what SECoMo can contribute to the different areas of approaches and if it can possibly even enhance some of them.
%REMINDER: SECoMo represents a concrete estimation approach for the impact a software system has regarding ecological sustainability during its usage phase. 

\paragraph{SECoMo and Sustainability Metrics}
% with metrics / AS metrics

%Maybe in contribution part -> this is new?! cause not much about "`estimation"' yet (true??)

\paragraph{SECoMo in SDLC(?) models}
%REFER back to the part how to generally include it? or move it here?
% Find / create some ideas - where to integrate it in some models
% - in lifecycle models

%here: earlier = cheaper aspect + where mentioned? in impacts paper!


\paragraph{SECoMo and Requirements Engineering}
%in relation with requirement topics?! -> later!!

% + is also very nice for the problem of NFR -> ?! see chapter again



%%% Further aspects
% for eco-label?!
% !!! Integrate with GREEN-IT readiness / maturity models!?


\subsection{Limitations}
% of secomo - what it still does not contribute - social sust. (more a process related aspect?)
%+ in general: social sust. -> still not much work done in general (see paper Hinai & CHit..) -> and also from my list -> most focus on environmental sustainability, so more an issue in the field? 

%in order to see how it contributes to having more practical evaluations - only guessing? - needs to be seen

%calibration models!? a) hard??? ungenau b) only implicitly covers impact of programming style, languages, libraries used, dev environments etc -> there could be a great deal of influence (see noureddine -> impact of programming language? and capra -> influence of dev environment?)

%of the integration - how it can't be perfectly integrated?!

%\chapter{Future work}
%ideas like - include secomo in a "real" project -> how does it work?  
%how realistic is the "calibration" process, or too much overhead?

