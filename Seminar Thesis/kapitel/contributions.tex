%Multiple sections, based on what I do
\chapter{Contributions}
%\section{Why sustainable software engineering is not yet used in practice(?!)} %see results of SLR?! + paper of Eva Kern
% of existing models / methods / metrics:
% critique - what is missing? - not used in practice?! (quellen?!) - why?

\section{SeCoMo in the context of Sustainable SW Engineering}
%What is done in the thesis?
%%%%Einordnung
\paragraph{Classification}
%how does secomo fit IN Sust. SE - relate it to DEFINITION !

% How do they relate? What does it contribute to sustainability?
%-> fits in: system usage aspect of penzenstadler categorization



\paragraph{SECoMo and Sustainability Metrics}
% with metrics / AS metrics


\paragraph{SECoMo in SDLC(?) models}
%REFER back to the part how to generally include it? or move it here?
% Find / create some ideas - where to integrate it in some models
% - in lifecycle models


\paragraph{SECoMo and Requirements Engineering}
%in relation with requirement topics?! -> later!!

% + is also very nice for the problem of NFR -> ?! see chapter again




% for eco-label?!
% !!! Integrate with GREEN-IT readiness / maturity models!?


%\chapter{Findings(?)} %or above part here... - AND: separate from Conclusion?!
% What is enhanced? Answer the question here...


\subsection{Limitations}
% of secomo - what it still does not contribute - social sust. (more a process related aspect?)
%+ in general: social sust. -> still not much work done in general (see paper Hinai & CHit..) -> and also from my list -> most focus on environmental sustainability, so more an issue in the field? 

%in order to see how it contributes to having more practical evaluations - only guessing? - needs to be seen

%calibration models!? a) hard??? ungenau b) only implicitly covers impact of programming style, languages, libraries used, dev environments etc -> there could be a great deal of influence (see noureddine -> impact of programming language? and capra -> influence of dev environment?)

%of the integration - how it can't be perfectly integrated?!

%\chapter{Future work}
%ideas like - include secomo in a "real" project -> how does it work?  
%how realistic is the "calibration" process, or too much overhead?

