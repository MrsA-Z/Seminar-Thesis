%The main contributions / findings of the work
\section{Conclusion}
In the context of Sustainability, ICT plays a major role as contributor to negative impacts, but also as enabler to improve Sustainability - in the ICT sector itself, but also in other sectors. Even though it might not be obvious at the first glance, software products are relevant in that context, too: They can have direct and indirect impacts on the different dimensions of Sustainability, for example by influencing a hardware product's resource consumption or by promoting sustainable life styles.\\
The research field of sustainable software engineering deals with concepts and approaches on how to ensure that software products can improve their positive impact on Sustainability during their whole life cycle. Over the past few years, the amount of contributions to the field has significantly increased, and a variety of concepts exist that tackle different aspects of software sustainability. But a common understanding and concrete guidance is still missing in the field, and despite the variety of approaches, not much is evaluated in practice, yet.

The SECoMo approach introduced in this paper is yet another addition to the contributions of this field, in particular in the area of approaches dealing with software usage sustainability. %something more about its classification?
It has the ability to enhance existing approaches in the field in various ways and might thus be suitable to help leading Sustainable Software Engineering to more adaption in real life projects. In any way, SECoMo can play an important role in the field in different areas:\\
With regards to existing metrics, SECoMo provides a new set of eco-cost metrics that can be applied to various types of software development projects as they are more flexible and can cover a broad range of contexts.\\
In comparison to existing sustainability models in the field, an important role that SECoMo can play is to enhance the already well-rounded GREENSOFT model with a concrete estimation approach for environmental sustainability aspects in order to make its development procedure model more likely to be adapted in practice. Over all, SECoMo introduces a new type of approach to Sustainable Software Engineering with its ability to estimate eco-costs even before development phase has started.\\
SECoMo can even contribute to the important area of Requirements Engineering by providing a much needed vocabulary to define sustainability requirements and also the respective method how to assess them. %%%%Abstract: As SECoMo can be integrated in all development phases, especially the early ones, it can help to enhance existing life cycle models with a concrete method for understanding and improving ecological sustainability in the design and implementation phases of software engineering. In addition, with its new set of sustainability metrics, SECoMo offers new options for sustainability measurement and assessment in existing models and tools, which base on a general way of software specification that can reasonably be applied in practice.
%can it help in practice?
%limitations?

The future work on SECoMo will show whether it is indeed a suitable approach that can enable the adoption of Sustainable Software Engineering in practice, and in any way there is still some work to do in the research field itself, starting from finding a common understanding of the underlying concepts over to evaluating the existing variety of approaches in real life projects. But the trend of Green IT and Sustainability adaption in companies for example shows, that the work in this direction is important and very much needed.
%ausblick - secomo - needs to be tested in practice -> and in combination! to say more about practice - but good starting point as very good for all kinds of sw projects but also very adaptive for different needs of stakeholders
%LATER also: more work for other dimensions of sutainbility?! - even though according to penzenstadler for example, every other of her 5 dimensions is more covered than environmentall? whatever