\chapter{Introduction}
%Introduction into the topic
Under the slogan "Sustainable Development Goals - 17 goals to transform our world"\cite{nino_sustainable_2017}, the United Nations are raising awareness for Sustainability and the need for everybody's action in order to "`shift the world on to a sustainable and resilient path"' \cite{UN_transform_15}%cite p. 1???
. The 17 goals mentioned are a central part of the sustainable development agenda published by the United Nation in 2015 \cite{UN_transform_15}%cite or just link?)
 and are essential in order to achieve sustainability with the main topics of "`end[ing] poverty, protect[ing] the planet and ensur[ing] prosperitiy for all"'\cite{nino_sustainable_2017}.\\
These ambitious goals emphasize how important the topic of Sustainability has become for the world, which is facing global challenges such as extreme poverty, climate change or war and injustices. Despite the global relevance of these issues, the scope of these goals reveals that taking action for sustainability affects everybody: "`governments, the private sector, civil society"'\cite{nino_sustainable_2017} and every private person. 
%short - what is sustainability - when did it first come up (bruntland?) - 3 dimensions - but still has lost nothing of its (aktualität) - some examples (klimaschutzgipfel, thema trump? :D denying global warming? - soziale aspekte, wirtschaftliche aspekte) - not TOO much

This is especially true when considering the causes for these challenges, spanning for example corrupt governments that cause injustice in their countries, or the ever increasing pollution and carbon dioxide emissions caused by various industries in the private sector. Many individuals and groups % / organizations?? %??? -> see starting argumentation of "`The role of corporations"' - LUnd
 have a negative impact on sustainability with their actions - but on the other hand, there is also a great potential for positive impacts, starting from there.\\
 Embracing sustainability has become more important in many areas over the last decades %do I need to cite something here? e.g. un reports when did they start to consider that topic? 1992 - rio?!
. Accordingly, many industries in the private sector have started to consider how to reduce their impacts on sustainability, often with a focus on ecological sustainability. %recognized their responsibility!
% Contributors to negative influence - different (branchen?), with production / manufacturing and related ecological impacts - or organizations working to fight poverty or social injustice - or ...

%-SHORTEN!!!!
The ICT sector is no exception: with the products and services it provides it contributes to high power consumption and carbon dioxide emission %include gartner number here: 2007 - 2 % of overall global carbon emissions
 and thus has a negative effect on global warming \cite{smart2020} %hardware and software
- but in the last few years, many efforts have been made to counteract these negative impacts, for example by improving energy efficiency of hardware products or by creating software solutions that support more eco-friendly business processes.\\
% IT sector - no different, also involved: in negative AND positive ways (dual aspect - see grenn in SE paper) - with production of HW and USE -> mainly eco sustainability :( but with products, e.g. software - can support projects that work to improve sustainabilty (all aspects) ! 
% + reducing negative impacts -> main task! 
However, the focus of these efforts is still mainly on reducing the negative impacts of hardware artifacts, %or products???
 during their production and, more importantly, usage phase. But software artifacts can have an impact on sustainability issues, too - not only by enabling people or processes using it to be more sustainable, but also as a product itself, which for example impacts energy consumption during usage, or even during its development process. %references for that????? keep in mind! 
% 2 focus areas: Hardware - production and use!
% but also software -> (see abstract) why? because software INFLUENCEs how the hw is used and can thus indirectly change the hw impact + when it is produced itself, there are improvements possible (e.g. also indirectly, by allowing home-office :D)
Thus, efforts directed at improving software sustainability during its development and usage phases are equally important in terms of IT Sustainability. And the research area of Sustainable Software Engineering has actually /indeed %%%????
 started to grow % word???? increase? expand?
significantly over the past five to six years. %reference to SLR study / SMS?
 %%%%% NOW: an overview about what is there?
% like: 
There is a growing number of papers %????
covering ideas for example on sustainability guidelines for software, models for sustainable development life cycle models, abstract software sustainability metrics or specific measurement tools for certain types of applications. %less specific - in that detail only in the actual part

% but issues -> (see abstract) 
%.. and re-work later when paper is finished!!!!!!!
What is lacking %not yet much available // rare???
are %%experiences? reports??
concrete applications of these guidelines, models and metrics in practical projects %real-life projects??
and evaluations of the validity and practicality of these various approaches.
Reasons for that might be the %neuheitß??
fact that it is a research field that has only emerged during the past couple of years and is still in its first steps %seee paper of SLR etc.
. Another aspects though might be that there are still more concrete approaches and metrics missing, that are generic enough to be applied to various software engineering projects, but specific enough to be directly used in practice.
%% like abstract: with concrete methods of calculating and reducing ecological costs missing.\\


%purpose - see how existing models can be enhanced by new method SECoMo (no description here???) - to work on some of these aspects
%description of work done (more detailed)
%(... später, wenn auch tatsächlich was geschrieben wurde...)
This paper aims at comparing a new approach in the field of Sustainable Software Engineering, SECoMo by Thomas Schulze (2016) %correct citation!!
, with the existing variety of models and approaches in the field. The goal %?????
is to work out how this approach for estimating the ecological costs of a software system can enhance existing approaches, models or metrics %%%(?)or the field itself
in order to enable sustainable software engineering to be more likely to be applied in practical projects. %%mutig...


%Structure of the thesis
The relevant background topics for this seminar thesis, Sustainability and the relationship %????
between Sustainability and ICT / IT, %?????
are introduced in the first chapters including definitions. %????
In the following, the related work from the field of Sustainable Software Engineering is described, covering the %most relevant?
most important existing guidelines, methods, models and metrics. %adapt lateron!!
In addition, the SECoMo approach is introduced %alternative word
and described with a focus on its general purpose and motivation as well as its most relevant elements and how to apply SECoMo within a software development process. %??? can I say that? is that okay? it's kind of my "`own"' work -> @ atkinson
\\ Based on these foundations, SECoMo is then considered in comparison to %some relevant examples of the existing modesl etc.???
existing methods and approaches in order to identify benefits of this approach for the field of Sustainable Software Engineering.
%Lastly, .... 
%for example: Secomo is also considered together with some other, "`rand"'-themen, e.g. eco-label, education, etc. to see further benefits!




%\chapter{Schriftliche Ausarbeitung}
%
%\section{Formale Anforderungen}
%
%Der nachstehende Abschnitt gibt einen kurzen Überblick über die
%formalen Anforderungen an die schriftliche Ausarbeitung von Seminar-,
%Studien-, Bachelor-Abschluss- und Diplomarbeiten. Im Weiteren bitten wir
%unbedingt, die für Sie gültige Prüfungsordnung zu beachten.
%
%Für alle Ausarbeitungen sind die Templates des Lehrstuhls (\url{http://swt.informatik.uni-mannheim.de/de/studies/master-and-bachelor-theses/guidelines/}) zu verwenden.
%
%\begin{table}[tb]
%\centering
%\begin{tabular}{|p{0.35\textwidth}*{1}{|p{0.25\textwidth}}|p{0.25\textwidth}|}
%\hline
%& \textbf{Bachelorarbeit} & \textbf{Masterarbeit}\\
%\hline
%Seitenzahl (ohne Anhang) & ca. 30 & ca. 80\\ 
%\hline
%Abbildungsverzeichnis & ja & ja\\
%\hline
%Tabellenverzeichnis & ja & ja\\
%\hline
%Ehrenwörtliche Erklärung & ja & ja\\
%\hline
%Abtretungserkl{\"a}rung & nach Vereinbarung & nach Vereinbarung\\
%\hline
%Bearbeitungszeit & max. 3 Monate & max. 6 Monate\\
%\hline
%Anzahl der abzugebenden Examplare (+ CD) & 3 Exemplare (gebunden) an LS Sekretariat & 3 Exemplare (gebunden) an LS Sekretariat\\
%\hline
%\end{tabular}
%\caption{Übersicht der Anforderungen \label{tableAnforderungen}}
%\end{table}
%
%Bei Abweichungen von diesen formalen Anforderungen, insbesondere bei der Seitenanzahl, ist unbedingt Rücksprache mit dem zuständigen Betreuer zu halten. Bei der Anfertigung der Arbeit sind die Grundregeln des wissenschaftlichen Arbeitens zu beachten. Zum Beispiel sind Quellen anzugeben, wörtlich aus Quellen übernommene Textstellen als Zitate zu kennzeichnen usw.
%
%\subsection{Verwendete Literatur}
%
%Gute Literatur zur Bearbeitung der Arbeit kann unter folgenden Adressen gefunden werden:
%
%\begin{itemize}
%\item IEEE Computer Society Digital Library: \url{http://www.computer.org/portal/site/csdl/index.jsp}
%\item ACM Digital Library: \url{http://portal.acm.org/dl.cfm}
%\item CEUR Workshop Proceedings: \url{http://ceur-ws.org/}
%\item Springer Link: \url{http://www.springerlink.com/home/main.mpx}
%\end{itemize}
%
%Die Dokumente von ACM als auch Springer sind mit einer universitären IP (z.B. über VPN-Verbindung oder aus dem Pi-Pool) kostenlos zu bekommen. Zu IEEE besitzt die Universitätsbibliothek Zugangsdaten. CEUR dagegen enthält nur frei zugängliche Dokumente.
%
%Das Literaturverzeichnis ist enstprechend den Vorgaben des Templates aufzubauen. Ein Beispiel f\"ur eine Zitation: \cite{tannenbaum}.
%
%\subsection{Inhalt}
%
%Interessant Hinweise zum Erstellen einer Forschungsarbeit liefert z.B. \url{http://www.cs.cmu.edu/~Compose/shaw-icse03.pdf}
%
%\subsection{Abgabe}
%
%Die schriftlichen Ausfertigungen müssen fristgerecht im Sekretariat oder beim Betreuer abgegeben werden. Ansonsten kann nach den Bestimmungen der Prüfungsordnung keine erfolgreiche Leistung bescheinigt werden. 
%
%Jede Arbeit soll doppelseitig gedruckt werden und dem Layout eines guten
%Buches entsprechen. Wie in der Tabelle oben ersichtlich, benötigen werden drei gebundene Exemplare. Alle Exemplare der Bachelor-Abschluss- bzw. Diplomarbeit müssen die ehrenwörtliche Erklärung unterschrieben beinhalten. Auf der inneren Rückseite von jedem Exemplar ist eine eingebundene/eingeklebte Hülle mit einer CD mit abzugeben. Der Inhalt dieser CD:
%
%\begin{itemize}
%\item letzte Fassung der Arbeit (auch in veränderlicher Form, z.B. *.doc, *.tex)
%\item Foliensatz des Abschlussvortrags (auch in veränderlicher Form, z.B. *.ppt, *.tex)
%\item Quellcode der programmierten Anwendung
%\item ausführbares Programm (und eine Anleitung, wie dieses zu verwenden ist)
%\item Dokumentation zu dem Programm (z.B. Javadoc)
%\item Referenzen (in der Arbeit verwendete Dokumente, die in elektronischer Form vorliegen)
%\end{itemize}
%
%Der Schein f{ür die Seminararbeit bzw. der erfolgreiche Abschluss der Studien-, Bachelor-Abschluss- bzw. Diplomarbeit wird dem Studienbüro erst gemeldet, wenn die Arbeit ordnungsgemäß abgegeben und der Vortrag absolviert wurde. 
%
%Eventuell ausgehändigte Schlüssel, Bücher oder sonstige Leihgaben des Lehrstuhls müssen bei der Abgabe der schriftlichen Ausfertigung beim Entleiher bzw. im Sekretariat zurückgegeben werden.
%
%\subsection{Abschlussvortrag}
%
%Kandidat/in und Betreuer vereinbaren einen Termin für den Vortrag, der in Anwesenheit von Herrn Prof. Atkinson gehalten wird. Die Folien sind dem Betreuer eine Woche vorher zuzusenden. Eine wichtige Fähigkeit bei Präsentationen (z.B. bei Konferenzen, berufliche Präsentationen) ist, Inhalte in begrenzter Zeit auf den Punkt zu bringen. Daher sollte die Vortragsdauer unbedingt eingehalten werden (Nichteinhalten kann zu einer schlechteren Benotung führen).
%
%\begin{table}[h]
%\centering
%\begin{tabular}{|l|c|c|}
%\hline
%& \textbf{Bachelor} & {\textbf{Masterarbeit}}\\
%\hline
%Vortragsdauer & 15min & 20min\\
%\hline
%Diskussion & 5min & 10min\\
%\hline
%\end{tabular}
%\caption{Vortragsdauer \label{tableVortragsdauer}}
%\end{table}
%
%%In this section, there is a first picture \ref{flow1}.
%
%\begin{figure}[h]
%\centering
%\includegraphics[width=0.07\textwidth]{grafiken/grafik}
%\caption{Beispielgraphik}
%\label{flow1}
%\end{figure}
