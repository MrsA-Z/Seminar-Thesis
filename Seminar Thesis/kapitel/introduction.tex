\chapter{Introduction}
%Introduction into the topic
%s. Abstract, nur ausführlicher?!

%description of work done (more detailed)
%(... später, wenn auch tatsächlich was geschrieben wurde...)

%Structure of the thesis
%(...)





%\chapter{Schriftliche Ausarbeitung}
%
%\section{Formale Anforderungen}
%
%Der nachstehende Abschnitt gibt einen kurzen Überblick über die
%formalen Anforderungen an die schriftliche Ausarbeitung von Seminar-,
%Studien-, Bachelor-Abschluss- und Diplomarbeiten. Im Weiteren bitten wir
%unbedingt, die für Sie gültige Prüfungsordnung zu beachten.
%
%Für alle Ausarbeitungen sind die Templates des Lehrstuhls (\url{http://swt.informatik.uni-mannheim.de/de/studies/master-and-bachelor-theses/guidelines/}) zu verwenden.
%
%\begin{table}[tb]
%\centering
%\begin{tabular}{|p{0.35\textwidth}*{1}{|p{0.25\textwidth}}|p{0.25\textwidth}|}
%\hline
%& \textbf{Bachelorarbeit} & \textbf{Masterarbeit}\\
%\hline
%Seitenzahl (ohne Anhang) & ca. 30 & ca. 80\\ 
%\hline
%Abbildungsverzeichnis & ja & ja\\
%\hline
%Tabellenverzeichnis & ja & ja\\
%\hline
%Ehrenwörtliche Erklärung & ja & ja\\
%\hline
%Abtretungserkl{\"a}rung & nach Vereinbarung & nach Vereinbarung\\
%\hline
%Bearbeitungszeit & max. 3 Monate & max. 6 Monate\\
%\hline
%Anzahl der abzugebenden Examplare (+ CD) & 3 Exemplare (gebunden) an LS Sekretariat & 3 Exemplare (gebunden) an LS Sekretariat\\
%\hline
%\end{tabular}
%\caption{Übersicht der Anforderungen \label{tableAnforderungen}}
%\end{table}
%
%Bei Abweichungen von diesen formalen Anforderungen, insbesondere bei der Seitenanzahl, ist unbedingt Rücksprache mit dem zuständigen Betreuer zu halten. Bei der Anfertigung der Arbeit sind die Grundregeln des wissenschaftlichen Arbeitens zu beachten. Zum Beispiel sind Quellen anzugeben, wörtlich aus Quellen übernommene Textstellen als Zitate zu kennzeichnen usw.
%
%\subsection{Verwendete Literatur}
%
%Gute Literatur zur Bearbeitung der Arbeit kann unter folgenden Adressen gefunden werden:
%
%\begin{itemize}
%\item IEEE Computer Society Digital Library: \url{http://www.computer.org/portal/site/csdl/index.jsp}
%\item ACM Digital Library: \url{http://portal.acm.org/dl.cfm}
%\item CEUR Workshop Proceedings: \url{http://ceur-ws.org/}
%\item Springer Link: \url{http://www.springerlink.com/home/main.mpx}
%\end{itemize}
%
%Die Dokumente von ACM als auch Springer sind mit einer universitären IP (z.B. über VPN-Verbindung oder aus dem Pi-Pool) kostenlos zu bekommen. Zu IEEE besitzt die Universitätsbibliothek Zugangsdaten. CEUR dagegen enthält nur frei zugängliche Dokumente.
%
%Das Literaturverzeichnis ist enstprechend den Vorgaben des Templates aufzubauen. Ein Beispiel f\"ur eine Zitation: \cite{tannenbaum}.
%
%\subsection{Inhalt}
%
%Interessant Hinweise zum Erstellen einer Forschungsarbeit liefert z.B. \url{http://www.cs.cmu.edu/~Compose/shaw-icse03.pdf}
%
%\subsection{Abgabe}
%
%Die schriftlichen Ausfertigungen müssen fristgerecht im Sekretariat oder beim Betreuer abgegeben werden. Ansonsten kann nach den Bestimmungen der Prüfungsordnung keine erfolgreiche Leistung bescheinigt werden. 
%
%Jede Arbeit soll doppelseitig gedruckt werden und dem Layout eines guten
%Buches entsprechen. Wie in der Tabelle oben ersichtlich, benötigen werden drei gebundene Exemplare. Alle Exemplare der Bachelor-Abschluss- bzw. Diplomarbeit müssen die ehrenwörtliche Erklärung unterschrieben beinhalten. Auf der inneren Rückseite von jedem Exemplar ist eine eingebundene/eingeklebte Hülle mit einer CD mit abzugeben. Der Inhalt dieser CD:
%
%\begin{itemize}
%\item letzte Fassung der Arbeit (auch in veränderlicher Form, z.B. *.doc, *.tex)
%\item Foliensatz des Abschlussvortrags (auch in veränderlicher Form, z.B. *.ppt, *.tex)
%\item Quellcode der programmierten Anwendung
%\item ausführbares Programm (und eine Anleitung, wie dieses zu verwenden ist)
%\item Dokumentation zu dem Programm (z.B. Javadoc)
%\item Referenzen (in der Arbeit verwendete Dokumente, die in elektronischer Form vorliegen)
%\end{itemize}
%
%Der Schein f{ür die Seminararbeit bzw. der erfolgreiche Abschluss der Studien-, Bachelor-Abschluss- bzw. Diplomarbeit wird dem Studienbüro erst gemeldet, wenn die Arbeit ordnungsgemäß abgegeben und der Vortrag absolviert wurde. 
%
%Eventuell ausgehändigte Schlüssel, Bücher oder sonstige Leihgaben des Lehrstuhls müssen bei der Abgabe der schriftlichen Ausfertigung beim Entleiher bzw. im Sekretariat zurückgegeben werden.
%
%\subsection{Abschlussvortrag}
%
%Kandidat/in und Betreuer vereinbaren einen Termin für den Vortrag, der in Anwesenheit von Herrn Prof. Atkinson gehalten wird. Die Folien sind dem Betreuer eine Woche vorher zuzusenden. Eine wichtige Fähigkeit bei Präsentationen (z.B. bei Konferenzen, berufliche Präsentationen) ist, Inhalte in begrenzter Zeit auf den Punkt zu bringen. Daher sollte die Vortragsdauer unbedingt eingehalten werden (Nichteinhalten kann zu einer schlechteren Benotung führen).
%
%\begin{table}[h]
%\centering
%\begin{tabular}{|l|c|c|}
%\hline
%& \textbf{Bachelor} & {\textbf{Masterarbeit}}\\
%\hline
%Vortragsdauer & 15min & 20min\\
%\hline
%Diskussion & 5min & 10min\\
%\hline
%\end{tabular}
%\caption{Vortragsdauer \label{tableVortragsdauer}}
%\end{table}
%
%%In this section, there is a first picture \ref{flow1}.
%
%\begin{figure}[h]
%\centering
%\includegraphics[width=0.07\textwidth]{grafiken/grafik}
%\caption{Beispielgraphik}
%\label{flow1}
%\end{figure}
