\chapter{Introduction}
%Introduction into the topic
Under the slogan "Sustainable Development Goals - 17 goals to transform our world"\cite{nino_sustainable_2017}, the United Nations are raising awareness for Sustainability and the need for everybody's action in order to "shift the world on to a sustainable and resilient path" \cite{UN_transform_15}%cite p. 1???
. The 17 goals mentioned are a central part of the sustainable development agenda published by the United Nation in 2015 \cite{UN_transform_15}%cite or just link?)
 and are essential in order to achieve sustainability with the main topics of "end[ing] poverty, protect[ing] the planet and ensur[ing] prosperitiy for all" \cite{nino_sustainable_2017}.\\
These ambitious goals emphasize how important the topic of Sustainability has become for the world, which is facing global challenges such as extreme poverty, climate change or war and injustices. Despite the global relevance of these issues, the scope of these goals reveals that taking action for sustainability affects everybody: "governments, the private sector, civil society"\cite{nino_sustainable_2017} and every private person. 

This is especially true when considering the causes for these challenges, spanning for example corrupt governments that cause injustice in their countries, or the ever increasing pollution and carbon dioxide emissions caused by various industries in the private sector. Many individuals and groups % / organizations?? %??? -> see starting argumentation of "`The role of corporations"' - LUnd
 have a negative impact on sustainability with their actions - but on the other hand, there is also a great potential for positive impacts, starting from there.\\
 Embracing sustainability has become more important in many areas over the last decades %do I need to cite something here? e.g. un reports when did they start to consider that topic? 1992 - rio?!
. Accordingly, many industries in the private sector have started to consider how to reduce their impacts on sustainability, often with a focus on ecological sustainability. %recognized/taken on their responsibility!
% Contributors to negative influence - different (branchen?), with production / manufacturing and related ecological impacts - or organizations working to fight poverty or social injustice - or ...

%-SHORTEN!!!!
The ICT sector is no exception: with the products and services it provides it contributes to high power consumption and carbon dioxide emission %include gartner number here: 2007 - 2 % of overall global carbon emissions
 and thus has a negative effect on global warming \cite{smart2020} %hardware and software
- but in the last few years, many efforts have been made to counteract these negative impacts, for example by improving energy efficiency of hardware products or by creating software solutions that support more eco-friendly business processes.\\
% IT sector - no different, also involved: in negative AND positive ways (dual aspect - see grenn in SE paper) - with production of HW and USE -> mainly eco sustainability :( but with products, e.g. software - can support projects that work to improve sustainabilty (all aspects) ! 
% + reducing negative impacts -> main task! 
However, the focus of these efforts is still mainly on reducing the negative impacts of hardware artifacts, %or products???
 during their production and, more importantly, usage phase. But software artifacts can have an impact on sustainability issues, too - not only by enabling people or processes using it to be more sustainable, but also as a product itself, which for example impacts energy consumption during usage, or even during its development process. %references for that????? keep in mind! 
% 2 focus areas: Hardware - production and use!
% but also software -> (see abstract) why? because software INFLUENCEs how the hw is used and can thus indirectly change the hw impact + when it is produced itself, there are improvements possible (e.g. also indirectly, by allowing home-office :D)
Thus, efforts directed at improving software sustainability during its development and usage phases are equally important in terms of IT Sustainability.\\
 %%%%% NOW: an overview about what is there?
There is a growing number of scientific contributions covering ideas for example on sustainability guidelines for software, models for sustainable development life cycle models, abstract software sustainability metrics or specific measurement tools for certain types of applications. %less specific - in that detail only in the actual part

% but issues -> (see abstract) 
%.. and re-work later when paper is finished!!!!!!!
What is lacking %not yet much available // rare???
are %%experiences? reports??
concrete applications of these guidelines, models and metrics in practical projects %real-life projects??
and evaluations of the validity and practicality of these various approaches.
%Reasons for that might be the fact that it is a research field that has only emerged during the past couple of years and is still in its first steps. %seee paper of SLR etc.
% Another aspects though might be that there are still more concrete approaches and metrics missing, that are generic enough to be applied to various software engineering projects, but specific enough to be directly used in practice.
%% like abstract: with concrete methods of calculating and reducing ecological costs missing.\\


%purpose - see how existing models can be enhanced by new method SECoMo (no description here???) - to work on some of these aspects
%description of work done (more detailed)
%(... später, wenn auch tatsächlich was geschrieben wurde...)
This paper aims at comparing a new approach in the field of Sustainable Software Engineering, SECoMo by Thomas Schulze \cite{schulze_cost_2016}, with the existing variety of models and approaches in the field. The goal is to work out how this approach for estimating the ecological costs of a software system can enhance existing approaches for sustainable software and its development. %in order to enable sustainable software engineering to be more likely to be applied in practical projects. %%mutig...

%Structure of the thesis
The relevant background topics for this seminar thesis, Sustainability and the relationship between Sustainability and ICT,
are introduced and defined in the first chapter. In the following, the related work from the field of Sustainable Software Engineering is described, covering the current state of research and the most important existing approaches. %adapt lateron!!
 In addition, the SECoMo approach is introduced %alternative word
and described with a focus on its general purpose and motivation as well as its most relevant elements and how to apply SECoMo within a software development process. %??? can I say that? is that okay? it's kind of my "`own"' work -> @ atkinson
\\ In the next section, SECoMo is then considered in the context of Sustainable Software Engineering and analyzed regarding its ability to enhance existing approaches in this field, in order to identify its benefits for the field of Sustainable Software Engineering.\\
Lastly, limitations of SECoMo in the context of Sustainable Software Engineering are presented and a conclusion of the paper is presented.

%for example: Secomo is also considered together with some other, "`rand"'-themen, e.g. eco-label, education, etc. to see further benefits!
