\chapter{Background} %or Foundations?!
%What is used in the thesis e.g. formalisms, methodologies, ... (@ atkinson - any formalism / methodologies needed?!)
%Clearly separate between contributions (by me) and foundations (from others)
%Only describe what is needed to understand the thesis
\section{Sustainability}
Although it is clear that Sustainability is a concept of increasing importance in the world and it has wide-ranging influences, on the way countries are lead, companies are strategically aligning themselves(?) or which causes societies are engaging for(?) for example, the definition of the term Sustainability itself is not as clear and not always unified / consistent / coherent?: %improve
There are various attempts at understanding and defining the concept of Sustainability, coming from different perspectives which influence the focus of the definition (e.g. the needs of humans or of nature \cite{gladwin_shifting_1995}) and thus it can be hard to find a common understanding \cite{jamieson_sustainability_1998}%(cited?? or just referenced ) 
. In order to successfully work towards a sustainable future, finding this common understanding is absolutely necessary, though \cite{jamieson_sustainability_1998}.\\
%The word sustainability itself... %origin of the word -> see this one paper from lund? // or "`tower of Babel"' paper, p.1
%--> Sustainability and Beyond (weak & strong sustainability) 

However, most attempts at defining the concept of Sustainability start with the definition of the concept of \textit{Sustainable Development} by the United Nations World Commission on Environment and Development in 1987, also known as the \textit{Brundtland Report} %(named after its chairperson Gro-Harlem Brundtland)
: "Sustainable development is development that meets the needs of the present without compromising the ability of future generations to meet their own needs"\cite{Brundtland1987} % p.41!
. In addition, the requirements for sustainable development according to \cite{Brundtland1987} are related to three different aspects: society, economy and environment. These three aspects are now widely accepted as "the three dimensions of sustainable development" \cite{UN_transform_15}.
%-> also see: triple bottom line? -> but rather company focus.. could be added if needed
%also see -> penzenstadler adds individual & technical dimension?! lateron..
%something about weak and strong sustainability? technology will solve everything or not?

Based on this understanding of what Sustainable Development is and which dimensions it covers, the concept of Sustainability can be defined as "a holistic concept that embraces environmental, social and economic factors which lead to a decent life for the current generation while maintaining natural, social and economic resources so that future generations are not limited in living the same decent life" \cite{buchner_sust_16}.% p.4
 This definition recognizes Sustainability as a concept that enables Sustainable Development in all its dimensions which a focus on human needs, but not limited to it.
%(so more like weak sust?) - see sust & beyond paper

%growing attention for sustainability - not double from introductioN?
%sustainability for companies..?! --> sustainability in production and processes (see introduction of lund paper!?)


\section{Sustainability and IT}  %Sustainability in ICT first?! 
%unterkapitel einbinden später!!



\chapter{Related work in Sustainable Software Engineering}
\section{state of research}
% RELATED WORK - What exists?
%first - describe how it is an interesting research field and many contributions over the last years - refer to SLR / SMS etc. - what is the content of most papers? 
%also already - what is missing? 

%MAYBE - start with tate / focus on "`economic"' sustainability etc. - s. other papers with literature review etc.
%then - describe newer developments! overall 
%e.g. general efforst
% requests to integrate sustainability as NFR!!! (see all sources for that) -> what is needed to reach that? 

%overview: and what does already exist? give some categories or sth. 

%now: give / present some examples /that will be relevant later or are just important to be mentioned
\subsection{Principles, Guidelines, Best practices}
% principles, design, guidelines etc. (...??????) karlskrona manifesto!?
% in which aspects do they occur - strategic level for company, in design of SW & architecture, 
% considered in requirements engineering - what needs to be fulfilled? what aspects occur? (see contributions of papers / etc.)

% GREEN SOFTWARE QUALITY?

\subsection{Approaches / different stages} %e.g. requirement engineering, design / architecture, development - or use / disposal
%green specifications!?

\subsection{Models and metrics} %focus on which phase?!
% mainly: what kind of general approaches exist / models?
% Greensoft - lifecycle model, impacts (but not - metrics!?) (based on albertao?!)
% ...
% Metrics -> Albertao -> in contribution part: compare & show how SECOMO makes this better

% - or rather in the contribution part? -> %critique - what is missing? - not used in practice?! (quellen?!) - why?

\subsection{case studies, concrete examples, practical adoption?}
%show that here is a lot missing?!

%%%%%THE SECOMO APPROACH%%%%%%%%% FOLLOWING!!!!!!