\chapter{Background} %or Foundations?!
%What is used in the thesis e.g. formalisms, methodologies, ... (@ atkinson - any formalism / methodologies needed?!)
%Clearly separate between contributions (by me) and foundations (from others)
%Only describe what is needed to understand the thesis

\section{Sustainability and Sustainable Software Engineering}
% Definitions 
% Sustainability, what is it, which dimensions are regarded?

%überleitung, nicht zu viel weil sollte in introduction drin sein!! - role of information technology in IT, and especially softwae
%Defintion: Sustainable Software & Sustainable SW Engineering

% s. old paper: Impacts of software - why is sustainable sw engineering important? and how to observe that

\section{Related work in Sustainable Software Engineering}
% RELATED WORK - What exists?
%first - describe how it is an interesting research field and many contributions over the last years

%MAYBE - start with tate / focus on "`economic"' sustainability etc. - s. other papers with literature review etc.
%then - describe newer developments! overall


%NOW - categorize them
\subsection{Principles and Manifestos}
% principles, design, guidelines etc. (...??????) karlskrona manifesto!?
% in which aspects do they occur - strategic level for company, in design of SW & architecture, 
% considered in requirements engineering - what needs to be fulfilled? what aspects occur? (see contributions of papers / etc.)

% GREEN SOFTWARE QUALITY?

\subsection{Approaches / different stages} %e.g. requirement engineering, design / architecture, development - or use / disposal
%green specifications!?

\subsection{Models and metrics} %focus on which phase?!
% mainly: what kind of general approaches exist / models?
% Greensoft - lifecycle model, impacts (but not - metrics!?)
% ...

% - or rather in the contribution part? -> %critique - what is missing? - not used in practice?! (quellen?!) - why?

\subsection{The SECoMo approach} % RELATED WORK / FOUNDATION!?
%see extra chapter