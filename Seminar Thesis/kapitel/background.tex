\chapter{Background} %or Foundations?!
%What is used in the thesis e.g. formalisms, methodologies, ... (@ atkinson - any formalism / methodologies needed?!)
%Clearly separate between contributions (by me) and foundations (from others)
%Only describe what is needed to understand the thesis
\section{Sustainability}
% Definitions % Sustainability, what is it, 
The term Sustainability ... %-> brundtland report

%which dimensions are regarded? UN -> 3, (+ human? -> nee, weg) -> also see: triple bottom line? -> but rather company focus.. could be added if needed
%dimensions also mentioned in the 2015 agenda (p. 1 unten)


\section{Sustainability and IT}  %Sustainability in ICT first?! 
As mentionend before %can I say that? "`before"'?
, the Information and Communication Technology (ICT) %abkürzungen????
sector/industry plays, as many industries do, an important role in achieving Sustainability. Even though this covers all three dimensions, the focus is often on ecological sustainability, leading to the frequent use of the term \textit{Green IT} covering this aspect of IT Sustainability. %reference to the calero / piattini paper?!
%green it -> focus often on Hardware! 

%abgrenzen: Green vs. Sustainable -> gleich benutzen oder unterschieldlich? Bezug nehmen auf calero / piattini paper
%abgrenzen (wo?) - green BY it and green (for) IT -> und übertragbar natürlich auch auf SUST. ?
% - role of information technology in IT, research in Green IT etc. - two components: Hardware AND software
%describe hardware efforts etc. shortly, idea behing it -> e.g. improving hardware components - or data centers of a company.. or or or -> many papers on that!?


%überleitung: und warum jetzt software -> hat das überhaupt einen einfluss? (see lund paper arguments ->)
% s. old paper: Impacts of software - why is sustainable sw engineering important? and how to observe that -> see Amsel / Argawal! 
%+ the first second, third order impacts etc.?!

%Defintion: Sustainable Software & Sustainable SW Engineering - abgrenzung zu "`green"' software engineering, wie das TRIER immer nennt?

%name different aspects, maybe those from penzenstadler - in the PROCESS aspect and in the PRODUCT aspect (could be refined later)


\chapter{Related work in Sustainable Software Engineering}
\section{state of research}
% RELATED WORK - What exists?
%first - describe how it is an interesting research field and many contributions over the last years - refer to SLR / SMS etc. - what is the content of most papers? 
%also already - what is missing? 

%MAYBE - start with tate / focus on "`economic"' sustainability etc. - s. other papers with literature review etc.
%then - describe newer developments! overall 
%e.g. general efforst
% requests to integrate sustainability as NFR!!! (see all sources for that) -> what is needed to reach that? 

%overview: and what does already exist? give some categories or sth. 

%now: give / present some examples /that will be relevant later or are just important to be mentioned
\subsection{Principles, Guidelines, Best practices}
% principles, design, guidelines etc. (...??????) karlskrona manifesto!?
% in which aspects do they occur - strategic level for company, in design of SW & architecture, 
% considered in requirements engineering - what needs to be fulfilled? what aspects occur? (see contributions of papers / etc.)

% GREEN SOFTWARE QUALITY?

\subsection{Approaches / different stages} %e.g. requirement engineering, design / architecture, development - or use / disposal
%green specifications!?

\subsection{Models and metrics} %focus on which phase?!
% mainly: what kind of general approaches exist / models?
% Greensoft - lifecycle model, impacts (but not - metrics!?) (based on albertao?!)
% ...
% Metrics -> Albertao -> in contribution part: compare & show how SECOMO makes this better

% - or rather in the contribution part? -> %critique - what is missing? - not used in practice?! (quellen?!) - why?

\subsection{case studies, concrete examples, practical adoption?}
%show that here is a lot missing?!

%%%%%THE SECOMO APPROACH%%%%%%%%% FOLLOWING!!!!!!